\documentclass[Main.tex]{subfiles} 
\begin{document}

\subsection{Definitioner}
\begin{tabular}{ l p{10cm} }
  \textbf{Accepttestspecifikation} & Dokument som specificerer test af funktionelle krav fra kravspecifikationen. Specifikation godkendes p� side 2 i henhold til kvalitetsplanen. \\\\
  \textbf{Accepttestrapport} & I udfyldt stand vil accepttesten udg�re en rapport. Rapporten godkendes i afsnit 4. \\\\
  \textbf{Internt testobjekt} & De objekter/testemner der er omfattet af denne accepttest. \\\\
  \textbf{Eksternt testobjekt} & Objekt der anvendes for at kunne udf�re testen, men som ikke er omfattet af godkendelse af accepttesten. En defekt fundet i et testobjekt vil s�ledes ikke umiddelbart kunne medf�re underkendelse af accepttesten. \\\\
  \textbf{Komplekse ruter} & Busruter som sp�nder mere end to endestationer.\\\\
  \textbf{Simple ruter} & Busruter som kun sp�nder to endestationer.\\\\
  \textbf{Byruter}	& Busruter som k�rer igennem storbystruktur. Her menes der ensrettede gader, overlap af bussens egen rute samt korte gadestr�kninger.\\\\
  \textbf{F�rste og sidste stoppested} & I databasen bliver stoppestederne lagt p� ruten i den r�kkef�lge de tilf�jes i, i administrator v�rkt�jet. F�rste stoppested er den f�rste endestation, sidste stoppested er den sidste.\\\\
  \textbf{Eclipse}& Brugt IDE for udvikling af smartphone applikation.
\end{tabular}


\end{document}