\documentclass[Main.tex]{subfiles} 
\begin{document}

\subsubsection{Use Case 5: Test program}
\textbf{Test Case: Normalforl�b}\\
\begin{tabular}{l p{12cm}} 
FORBEREDELSE:	&	Det foruds�ttes at brugeren er logget ind i systemet som 'Programm�r'. Der skal v�re et program ved navn 'Testprogram'\footnote{Testprogrammet �bner og lukker robottens gripper} tilg�ngeligt. Robotten skal v�re homed.
\\
BESKRIVELSE:	&	Der testes at det er muligt at k�re en simulering af et brugerdefineret program. 
 \\ 
\end{tabular}
	\begin{testcase}
	\punkt 1-2
	\Aktion Programm�ren trykker p� 'Run' efterfulgt at 'Simulated robot'.
	\Forventet Systemet lister programmet 'Testprogram'.

	\punkt 3-5
	\Aktion Simuleringen af programmet startes ved tryk p� 'Start program'.
	\Forventet Informationer om programmets k�rsel vises p� sk�rmen, herunder en besked om at robotarmen er blevet flyttet.  

	\punkt 6
	\Aktion -
	\Forventet Programm�ren modtager en besked om, at simuleringen er fuldf�rt.
	\end{testcase}
	\textbf{Afvigelsesforl�b - Test program }
\\
\begin{tabular}{l p{12cm}} 
FORBEREDELSE:	&	Brugeren skal v�re logget ind som programm�r, og v�re i gang med en simulering af programmet 'Testprogram'.
\\
BESKRIVELSE:	&	Det testes, at det er muligt at afbryde en simulering.
\\
\end{tabular}
\begin{testcase}
\punkt 1
\Aktion Brugeren trykker p� 'Stop'.
\Forventet Simuleringen stoppes.
\end{testcase}	
\textbf{Test Case: Ikke funktionelle krav}\\
\begin{tabular}{l p{12cm}} 
FORBEREDELSE:	&	Det foruds�ttes, at brugeren er logget ind som programm�r. Desuden skal der lige v�re k�rt en simulering af 'Testprogram'.
\\
BESKRIVELSE:	&	Det testes, at loggen gemmes med tidspunkt og loglevel.  
\end{tabular}
	\begin{testcase}
	\punkt 
	\Aktion Programm�ren g�r ind i 'Show log'-vinduet og v�lger loggen for den aktuelle dato.
	\Forventet Der verificeres visuelt, at loggen er udskrevet Tid-Loglevel-Besked.
	\end{testcase}
\end{document}


