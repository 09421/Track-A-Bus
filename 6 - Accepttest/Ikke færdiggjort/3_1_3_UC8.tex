\documentclass[Main.tex]{subfiles} 
\begin{document}

\subsubsection{Use Case 8: Rediger stoppested}
\textbf{Test Case: Normalforl�b A}
\\
\begin{tabular}{ l  p{12cm} }
FORBEREDELSE: & Administratoren er logget ind p� administrations hjemmesiden, samt st�r p� 'Rediger stoppested' siden.
\\
BESKRIVELSE: & Det testes at det er muligt at tilf�je ny stoppested til systemet.
\end{tabular}
\\
\begin{testcase}
\punkt 1
\Aktion Administratoren trykker et sted p� kortet, indskriver navnet "teststop" i feltet 'Stop Name', samt trykker p� knappen 'Save stop'.
\Forventet Stoppestedet bliver persisteret i databasen.
\end{testcase}
\textbf{Test Case: Normalforl�b B}
\\
\begin{tabular}{ l  p{12cm} }
FORBEREDELSE: & Administratoren er logget ind p� administrations hjemmesiden, samt st�r p� 'Rediger stoppested' siden. Normalforl�b A skal v�re gennemf�rt.
\\
BESKRIVELSE: & Det testes at det er muligt at �ndre et stoppested der findes i systemet.
\end{tabular}
\\
\begin{testcase}
\punkt 1
\Aktion Administratoren v�lger stoppestedet "teststop" fra listen af alle stoppesteder i systemet, �ndrer navnet til "teststop2" og �ndrer placeringen, samt trykker p� knappen 'Save changes'.
\Forventet �ndringer af stoppested bliver persisteret p� databasen.
\end{testcase}
\\
\newpage
\textbf{Test Case: Normalforl�b C}
\\
\begin{tabular}{ l  p{12cm} }
FORBEREDELSE: & Administratoren er logget ind p� administrations hjemmesiden, samt st�r p� 'Rediger stoppested' siden. Normalforl�b A skal v�re gennemf�rt.
\\
BESKRIVELSE: & Det testes at det er muligt at slette et stoppested fra systemet.
\end{tabular}
\\
\begin{testcase}
\punkt 1
\Aktion Administratoren v�lger stoppestedet "teststop" fra listen af alle stoppesteder i systemet, samt trykker p� knappen 'Delete stop'
\Forventet Det valgte stoppested bliver slettet fra databasen.
\end{testcase}

\textbf{Test Case: Undtagelsesforl�b A}
\\
\begin{tabular}{ l  p{12cm} }
FORBEREDELSE: & Brugeren er logget ind som 'Administrator' p� administrations hjemmesiden, st�r p� 'Rediger stoppested' siden, samt der ikke er forbindelse til internet.
\\
BESKRIVELSE: & Det testes at der kommer en fejlmeddelselse om at det ikke er muligt at gemme, da der ikke er adgang til internet.
\end{tabular}
\\
\begin{testcase}
\punkt 1
\Aktion Administratoren trykker et sted p� kortet, indskriver "stopfejl" i feltet 'Stop name', samt trykker p� knappen 'Save stop'.
\Forventet Administratoren bliver pr�senteret for en fejlmeddelelse, der beskriver at det ikke er muligt at persitere �ndringer.
\end{testcase}
\end{document}