\documentclass[Main.tex]{subfiles} 
\begin{document}

\subsubsection{Use Case 4: Rediger busruter i listen af favoriter}
\textbf{Test Case: Normalforl�b A}
\\\begin{tabular}{l p{11cm}} 
FORBEREDELSE:	&	TrackABus android mobil-applikationen skal v�re startet, samt st� ved listen af busruter, desuden skal der i databasen v�re gemt rutepunkter, busstoppesteder samt en bus, for en valgte busrute.
\\
BESKRIVELSE:	&	Der testes, at det er muligt at favorisere en busrute.
\\\\ 
\end{tabular}
\\
\begin{testcase}
\punkt 1
\Aktion Brugeren tilkendegiver overfor systemet at han �nsker at favorisere en givet busrute, ved tryk p� favoriserings knappen, i listen over alle busruter.
\Forventet favoriserings knappen skifter farve, samt data om busruten vil blive hentet fra databasen og persisteret p� SQLite databasen.
\punkt 2
\Aktion Brugeren vender tilbage til startsk�rmen for mobil-applikationen.
\Forventet Den favoriserede busrute kan ses p� listen over favoriserede busruter.
\end{testcase}
\\
\textbf{Test Case: Normalforl�b B}
\\\begin{tabular}{l p{11cm}} 
FORBEREDELSE:	&	TrackABus android mobil-applikationen skal v�re startet, samt st� ved listen af busruter, desuden skal en busrute v�re favoriseret.
\\
BESKRIVELSE:	&	Der testes, at det er muligt at fjerne favorisering af en busrute.
\\\\ 
\end{tabular}
\\
\begin{testcase}
\punkt 1
\Aktion Brugeren tilkendegiver overfor systemet at han �nsker at fjerne favorisering en givet busrute, ved tryk p� favoriserings knappen, i listen over alle busruter.
\Forventet favoriserings knappen skifter farve, samt data om busruten vil blive fjernet fra SQLite databasen.
\punkt 2
\Aktion Brugeren vender tilbage til startsk�rmen for mobil-applikationen.
\Forventet Den favoriserede busrute kan ikke l�ngere ses p� listen over favoriserede busruter.
\end{testcase}
\\
\textbf{Test Case: Undtagelsesforl�b 1}
\\ \begin{tabular}{l p{11cm}}
FORBEREDELSE:	&	 TrackABus android mobil-applikationen skal v�re startet, samt st� ved listen af busruter, desuden skal der ikke v�re forbindelse til databasen.\\
BESKRIVELSE:	&	Der testes, at der bliver vist en fejlmeddelelse hvis der ikke kan skabes forbindelse til databasen.  \\ 
\end{tabular}
\\\\
\begin{testcase}
\punkt 1
\Aktion Brugeren tilkendegiver overfor systemet at han �nsker at favorisere en givet busrute, ved tryk p� favoriserings knappen, i listen over alle busruter.
\Forventet En fejlmeddelse bliver vist p� sk�rmen, der beskriver det ikke er muligt at tilg� databasen.
\end{testcase}
\\
\textbf{Test Case: Undtagelsesforl�b 2}
\\ \begin{tabular}{l p{11cm}}
FORBEREDELSE:	&	 TrackABus android mobil-applikationen skal v�re startet, samt st� ved listen af busruter, desuden skal der i databasen v�re gemt rutepunkter, busstoppesteder samt en bus, for en valgte busrute.\\
BESKRIVELSE:	&	Der testes, at hvis indl�sning af data bliver annulleret, ved tryk p� telefonens tilbage-knap, bliver data for busruten hentet f�rdig, samt der returneres til der til listen af busruter.  \\ 
\end{tabular}
\\\\
\begin{testcase}
\punkt 1
\Aktion brugeren trykker p� telefonens tilbage-knap.
\Forventet Indl�sning af data fors�tter i baggrunden, samt der retuneres til startsk�rmen.
\end{testcase}
\\
\textbf{Test Case: Undtagelsesforl�b 3}
\\ \begin{tabular}{l p{11cm}}
FORBEREDELSE:	&	 TrackABus android mobil-applikationen skal v�re startet, samt st� ved listen af busruter, desuden skal der i databasen v�re gemt rutepunkter, busstoppesteder samt en bus, for en valgte busrute.\\
BESKRIVELSE:	&	Der testes, at hvis systemet g�r i dvale, ved tryk p� telefonens home-knap, bliver data for busruten hentet f�rdig i baggrunden.  \\ 
\end{tabular}
\\\\
\begin{testcase}
\punkt 1
\Aktion brugeren trykker p� telefonens home-knap.
\Forventet Indl�sning af data fors�tter i baggrunden.
\end{testcase}
\end{document}