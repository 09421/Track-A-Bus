\documentclass[Main.tex]{subfiles} 
\begin{document}

\subsubsection{Use Case 1.1}
\textbf{Test Case: Normalforl�b 1}
\\
\begin{tabular}{l p{11cm}} 
FORBEREDELSE:	&	Den simple rute er valgt fra listen af alle ruter, og kortet er vist. En Bus k�rer p� ruten\\
BESKRIVELSE:	&	Der testes, at det er muligt at m�le og veje en given klods.   \\ 
\end{tabular}
\\\\
\begin{testcase}
\punkt 1-11
\Aktion En sorteringssekvens startes, og k�res igennem indtil klodsen er m�lt og vejet.
\Forventet Det verificeres at siderne er blevet m�lt til 5*5*5CM ($\pm$ 0,5CM) og v�gten til 200g ($\pm$ 5g). Dette verificeres gennem loggen.
\end{testcase}
\\
\textbf{Test Case: Ikke-funktionelle krav 1}
\\
\begin{tabular}{l p{11cm}} 
FORBEREDELSE:	&	Systemet skal v�re sat til at k�re standardprogrammet. Den gr�nne klods' sider er m�lt pr�cist med en lineal, hvorefter den er placeret p� transportb�ndet. \\
BESKRIVELSE:	&	Der testes om robotarmen m�ler klodsens sider med en maksimum afvigelse p� $\pm$ 0.5 cm fra klodsens m�l.   \\ 
\end{tabular}
\\
\begin{testcase}
\punkt 
\Aktion Sorteringsprogrammet k�res og ved afslutning sammenlignes logviduets data vedr�rende siderne, med v�rdier m�lt med lineal.
\Forventet V�rdierne afviger mindre end $\pm$ 0.5 fra de med lineal m�lte v�rdier
\end{testcase}
\\
\textbf{Test Case: Ikke-funktionelle krav 2}
\\
\begin{tabular}{l p{11cm}} 
FORBEREDELSE:	&	Systemet skal v�re sat til at k�re standardprogrammet. Den gr�nne klods' v�gt er verificeret med en kalibreret v�gt, hvorefter klodsen er placeret p� transportb�ndet. \\
BESKRIVELSE:	&	Det testes om v�gten vejer klodsens v�gt med en maksimum afvigelse p� $\pm$ 5 gram.   \\
\end{tabular}
\\
\begin{testcase}
\punkt 
\Aktion Sorteringsprogrammet k�res og ved afslutning sammenlignes logviduets data vedr�rende v�gten med v�gten fra den kalibrerede v�gt.
\Forventet V�gten har en afvigelse mindre end $\pm$ 5 gram
\end{testcase}
\end{document}