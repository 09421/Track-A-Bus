\documentclass[Main.tex]{subfiles} 
\begin{document}

\subsubsection{Use Case 1: Vis busruter}
\textbf{Test Case: Normalforl�b A}
\\\begin{tabular}{l p{11cm}}
FORBEREDELSE:	&	TrackABus android mobil-applikationen skal v�re startet, samt st� ved startsk�rmen, desuden skal der i databasen v�re gemt en simpel rute, en kompleks rute, en byrute samt en test rute p� ca. 1 KM.
\\
BESKRIVELSE:	&	Der testes, at det er muligt at f� vist en liste af alle busruter gemt p� databasen.  
\\\\
\end{tabular}
\\
\begin{testcase}
\punkt 1
\Aktion Brugeren tilkendegiver overfor systemet, at han �nsker at f� vist alle gemte busrute, ved at at trykke p� knappen 'View busroutes'.
\Forventet De fire ruter, beskrevet i forberedelsen, vises.
\end{testcase}
\\
\textbf{Test Case: Undtagelsesforl�b A}
\\ \begin{tabular}{l p{11cm}}
FORBEREDELSE:	&	TrackABus android mobil-applikationen skal v�re startet, samt st� ved startsk�rmen, desuden skal der ikke v�re forbindelse til internetet \\
BESKRIVELSE:	&	Der testes, at der bliver vist en fejlmeddelelse hvis der ikke kan skabes forbindelse til databasen.  \\ 
\end{tabular}
\\\\
\begin{testcase}
\punkt 1
\Aktion Brugeren tilkendegiver overfor systemet, at han �nsker at f� vist alle gemte busrute, ved at at trykke p� knappen 'View busroutes'.
\Forventet En fejlmeddelse bliver vist p� sk�rmen, der beskriver, at det ikke er muligt at indl�se busruter.
\end{testcase}
\\\\
\textbf{Test Case: Undtagelsesforl�b B}
\\ \begin{tabular}{l p{11cm}}
FORBEREDELSE:	& 	Indl�sning af busruter er igangsat, samt der er gemt en simpel rute, en kompleks rute, en byrute samt en test rute p� ca. 1 KM, i databasen.  \\
BESKRIVELSE:	&	Der testes, at hvis indl�sning af busruter bliver annulleret, ved tryk p� telefonens tilbage-knap, returnerers der til applikationens startsk�rm.  \\
\end{tabular}
\\\\
\begin{testcase}
\punkt 1
\Aktion Brugeren annullerer indl�sningen af busruter
\Forventet Indl�sningen af busruter stoppes, samt der retuneres til applikationens startsk�rm.
\end{testcase}
\\
\textbf{Test Case: Undtagelsesforl�b C}
\\ \begin{tabular}{l p{11cm}}
FORBEREDELSE:	&	Indl�sning af busruter er igangsat, samt der er  gemt en simpel rute, en kompleks rute, en byrute samt en test rute p� ca. 1 KM, i databasen. \\
BESKRIVELSE:	&	Det testes, at hvis systemet g�r i dvale under indl�sning af busruter, ved tryk p� telefonens home-knap, vil systemet indl�se listen af busruter f�rdig i baggrunden.  \\
\end{tabular}
\\\\
\begin{testcase}
\punkt 1
\Aktion Brugeren trykker p� telefonens home-knap under indl�sning af busruter
\Forventet Systemet indl�ser busruterne f�rdig i baggrunder, ved gen�bning af programmet, vil listen af busruter kunne ses.
\end{testcase}
\end{document}

