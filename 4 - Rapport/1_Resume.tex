\documentclass[Main.tex]{subfiles} 
\begin{document}

\section{Resum�}
%Resum� (B�de dansk og engelsk)
%\\
%Resumeet er en appetitv�kker, der skal f� l�seren til at l�se resten af projektet. Det er en MEGET KORT (200-300 ord) sammenslutning af hele projektet, s� det er vigtigt at der ikke er detaljerede beskrivelser af projektet � dette sikrer ogs� at der ikke er for mange gentagelser i hele rapporten. 

I forbindelse med 4. semesters eksamensprojektet er der blevet udviklet styring til en robotarm, der tillader at sortere klodser efter materialetype. Systemet giver desuden brugeren mulighed for at udvikle simple programmer, og teste disse p� b�de den fysiske robot, s�vel som til en simuleret version af denne. For at sikre persistering af data, samt give brugeren mulighed for at gemme alle h�ndelser i systemet, er en relationel database blevet implementeret.
Det overordnede kodesprog i projektet er C\# og som udviklingsmilj� er Microsoft .NET framework blevet anvendt. 
Under .NET frameworket er WPF blevet brugt til, at udvikle den grafiske brugergr�nseflade. 
Som projektstyring er der bl.a. brugt dele af Scrum, hvor principper fra Extreme Programming og GRASP er blevet fundet brugbare med henhold til til kodeskrivning.
\\
Det er lykkedes at udvikle et program, der g�r robotten i stand til at sortere klodser efter materialetype. Derudover kan der via en udviklet IDE, skrives programmer til selvbestemte form�l. Desuden er programmet blevet udstyret med nogle ekstra funktioner, som giver brugeren en r�kke muligheder med henhold til ops�tning af systemet og manipulering med data.
\\
\\
In relation with the fourth semester project, a controlsystem to a robotic arm has been developed. The purpose of the system is to sort bricks in according to their material.  Furthermore the system holds a facility that allows the user to develop simple programs and test these on both the real robot as well as a simulation of the robot.  To guarantee persistence of the data and allow the user to save events from the system, a relational database has been implemented. The primary programming language used in this project is C\# and the development environment used was Microsoft's .NET framework. Within the .NET framework WPF has been used to develop the graphical user interface.  As project management method parts of Scrum have been used where principles from Extreme Programming and GRASP have showed useful when writing the source code.\\ 
A system which was able to sort bricks according to material type was implemented. Additional, an integrated development environment was developed which allows the programmer to create custom software for the robot. Furthermore functions  that gives the user opportunity to setup the system and manipulate the data have been added.
\end{document}