\documentclass[Main.tex]{subfiles} 
\begin{document}

\section{Resultater}
%OBJEKTIV vurdering af resultater. Meget gerne i tabelform, grafer og billeder. Der skal ikke v�re nogen form for diskussion i dette afsnit. Dette kommer i n�ste afsnit. 
Nedenfor er de v�sentligste resultater beskrevet. Der refereres til accepttestspecifikationen, hvis det �nskes at f� et st�rre overblik over de udf�rte tests.
\\\
\textbf{Overordnede resultater}
\begin{itemize}
\item F� en pr�cis tid til ankomsten for en bus ved et stoppested.
\item En m�de at administrere busruter.
\item Relationelle databaser til persistering af data.
\item Web service til database adgang fra mobil applikationen.
\item Simulering af busser.
\end{itemize}
\textbf{Mobil applikation}\\
Mobil applikationen kan pr�sentere brugeren for busruter, med dennes stoppesteder og k�rende busser. Det er samtidigt muligt at f� vist tiden til ankomst for en bus ved et valgt stoppested. Ruter kan desuden favoriseres, hvilket medf�rer at de gemmes lokalt.\\
\\
\textbf{Administrations v�rkt�j}\\
Det kan lade sig g�re, at administrere busruter. Dette indeb�rer at oprette, slette og �ndre ruter, stoppesteder og busser, samt hvilken rute de forskellige busser er tilknyttet.\\
\\
\textbf{Databaser}\\
Den distribuerede database er skabt med MySQL og best�r af otte tabeller. Disse indholder informationer ved�rende busruter, stoppesteder samt busser og deres GPS-koordinater.\\
Den lokale database er skabt med SQLite og best�r af fem tabeller. Disse indeholder informationer om busruter og stoppesteder og er s�ledes et udsnit af den distribuerede database. Denne oprettes sammen med mobil applikationen.
\\
\textbf{Web service}\\
Web servicen fungerer som et mellemled mellem mobil applikationen og MySQL databasen.
\\
\textbf{Simulator}\\
Det er muligt at simulere en eller flere busser der k�re p� deres tilknyttede busrute.\\
\end{document}