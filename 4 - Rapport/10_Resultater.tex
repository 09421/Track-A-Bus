\documentclass[Main.tex]{subfiles} 
\begin{document}

\section{Resultater}
%OBJEKTIV vurdering af resultater. Meget gerne i tabelform, grafer og billeder. Der skal ikke v�re nogen form for diskussion i dette afsnit. Dette kommer i n�ste afsnit. 
Nedenfor er de mest v�sentlige resultater listet, hvorefter de kort beskrives. Der refereres til accepttestspecifikationen, hvis det �nskes at f� et st�rre overblik over de udf�rte tests.
\\\
\textbf{Overordnede resultater}
\begin{itemize}
\item Sortering af klodser efter matrialetype via Scorbot-robotarm.
\item Implementering af IronPython som via scripting kan kontrollere robotarmen.
\item Interfacing af den udleverede strain gauge s�ledes, at signalet p� 0-100mV blev forst�rket til 0-5V.
\item Opbygning af en relationel database til persistent lagring af data samt alle informationer fra processen.
\item Opbygning af en grafisk brugergr�nseflade der kan styre systemet. 
\end{itemize}
\textbf{Sortering af klodser}\\
Robotarmen kan samle en klods op fra transportb�ndet, m�le alle sider, placerer klodsen p� v�gten, og veje den, udregne densiteten, og slutteligt placerer den i tilsvarende kasse.
\\\\
\textbf{IDE} \\
Det kan lade sig g�re, at skrive simple programmer til robotten ved brug af IronPython, og benytte et IntelliSense-lignende system til at autoudf�re kode.
\\\\
\textbf{V�gtcellen}\\
Den udleverede strain gauge gav et for svagt signal, der derfor skulle forst�rkes. Der blev designet et simpelt print, som kan forst�rke signalet til v�rdier, som interfacer med mega16 microcontrolleren. Der blev ogs� skrevet et program, som A/D konverterer signalet og sender det til PCen via en RS232-forbindelse.
\\\\
\textbf{Grafisk brugergr�nseflade}\\
Den grafiske brugergr�nseflade tager udgangspunkt i et login vindue, hvor man kan logge ind som operat�r eller programm�r. Derefter bliver man pr�senteret for en hovedmenu, hvori der kan �bnes forskellige vinduer med forskellige funktionaliteter. Det omhandler blandt andet, at man kan se log beskeder p� databasen, se en liste over de forskellig klodsers placering og starte/stoppe samt pause systemet i en given frekvens, fx mens den er i gang med at sortere en klods.
\\\\
\textbf{Database}\\
Den relationelle database er skabt med Microsoft SQL server og best�r af fem tabeller til at holde data om systemet. I systemet gemmes der en del persistente data, herunder brugere, klodser, positioner i en boks, systemevents og densiteter. Data bliver b�de gemt og hentet flere steder i systemet, og til det er der blevet implementeret et beskedk�-system som h�ndterer database tilgangen. Fra systemet side er det muligt at opdatere, hente fra, skrive til, og slette fra databasen. Desuden h�ndterer databasen selv vedligeholdelse af databasen med diverse triggers og identity kolonner.


\end{document}