\documentclass[Main.tex]{subfiles} 
\begin{document}

\section{Projektets fortr�ffeligheder}
Nedenfor er en r�kke funktionaliteter og arkitekturer i projektet, som gruppen er specielt stolt af, listet og beskrevet.
\\\\
\textbf{IDE - Specielt IntelliSense}\\
Der var lagt op til at IDE-delen af projektet ville blive meget tidskr�vende. Dog blev en l�sning fundet ved hj�lp af IronPython scriptsproget, som b�de sparede tid, samt resulterede i en god, funktionel l�sning. Den tid der blev sparet blev istedet brugt til at g�re IDE'en bedre, og der blev blandt andet implementeret IntelliSense, der kan foresl� valgmuligheder l�bende mens koden skrives. Selvom IDEen ikke altid fungerer helt fejlfrit, g�r den det ekstremt simpelt at skrive programmer.
\\\\
\textbf{GUI - MVVM} \\
I f�rste omgang blev brugergr�nsefladen udviklet uden nogen egentlig arkitektur, andet end forms and controls. Da gruppen blev introduceret til MVVM, stod det hurtigt klart at denne arkitektur m�tte implementeres. Det var ikke nogen nem opgave, da den eksisterende struktur var rodet, men ved h�rdt arbejde blev der implementeret MVVM samt et message system til at h�ndtere beskeder mellem ViewModel og Views.
\\\\
\textbf{Databasetilgang} \\
Det kan v�kke mange problemer at tilg� en database, og derfor blev der lagt meget arbejde i at udvikle en fornuftig implementering af denne tilgang. Derfor blev der lavet et message system baseret p� observer-pattern, der sikre at data gemmes tr�dsikkert, og uafh�ngigt af om der er forbindelse til databasen eller ej.
\\\\
\textbf{Interfacing til v�gtcellen}\\
Interfacing arbejdet til v�gtcellen var relativt hurtigt overst�et uden problemer, og igennem hele processen fungerede dette uden modifikationer. S�ledes har en har udviklingen af en virksom v�gt ikke kostet meget. Derfor betragtes den som en fortr�ffelighed, selvom den er simpelt udviklet. Der skal dog knyttes en kommentar til at en br�ndt IC gjorde v�gten upr�cis under accepttesten.

\end{document}