\documentclass[Main.tex]{subfiles} 
\begin{document}

\section{Resum�}
%Resum� (B�de dansk og engelsk)
%\\
%Resumeet er en appetitv�kker, der skal f� l�seren til at l�se resten af projektet. Det er en MEGET KORT (200-300 ord) sammenslutning af hele projektet, s� det er vigtigt at der ikke er detaljerede beskrivelser af projektet � dette sikrer ogs� at der ikke er for mange gentagelser i hele rapporten. 

I forbindelse med bachhelor projektet, er der blevet udarbejdet et system til at indl�se positionen for en bus, vise denne p� et kort, samt underette brugeren om, hvor lang tid der er til, at en bus er ved et givent stoppested. Desuden er der ogs� mulighed for, at ruter kan favoriseres, og hermed gemmes lokalt. Ud over bruger funktionaliterne, underst�ter systemet ogs� muligheden for, at en adminstrator kan oprette, vedligeholde og slette busser, busruter og stoppesteder.\\
Persisteringen af data sker i form af to relationelle databaser; En lokal og en distribueret. Den lokale giver brugeren mulighed for at gemme ruter der bruges ofte, og den distribuerede indeholder alt information om ruterne, samt bussernes nuv�rende og tidligere position. Den distribuerede database indeholder ogs� funktionalitet til at udregne tiden for en bus til et givet stoppested.\\
Den distruberede database er opbygget ved hj�lp af MySQL, og den lokale er opbygget ved hj�lp af SQLite. Brugeren tilg�r systemet igennem en Android platform, hvorigennem samtlige bruger-funktionaliteter kan tilg�s. Koodesproget brugt til dette er Java. Administrator hjemmesiden er opbygget ved hj�lp af HTML, CSS og JavaScript, men da den er bygget p� baggrund af ASP.NET er C\# det prim�re kodesprog i denne sammenh�ng. Til systemet er der blevet designet en simulator, som st�r for at simulere en bus, i alle henseende. Den er blevet udarbejdet i WPF ved brug af .NET framworket, med C\# som kodesprog.\\
Til projektstyring er der blevet brugt dele af Scrum, og herunder er V-modellen fundet yderst brugbar.\\
Det f�rdige produkt er et system, hvori en bruger kan holdes opdateret omkring busser placering, og hvori en administrator nemt kan udf�re vedligeholdelse.
In relation to the bachelor project, a system has been designed, to retrieve the position of a bus, draw it on a map, and notify the user of how much time there is, until the closest bus, is at the chosen stop. The user system also implements the functionality, that a route can be favoured, and saved locally. Besides this, the system also supports the possibility, that a administrator can create, maintain and delete buses, routes and stops.\\ 
The persistence of data, is done by two relational databases; One locally and one distributed. The local database makes it possible for the user to save a route that is used often, and the distributed contains all the information regard the routes, as well as the previous and current positions of the buses. The distributed database also contains the functionality to calculate the time until a bus reaches a specified stop.\\
The distributed database is created with MySQL and the local is created with SQLite. The user accesses the system through an Android platform, where all user-functionalities can be access through. The language used for this is Java. The administrator homepage, is created with HTML, CSS and JavaScript, but since it is created as an ASP.NET application, C\# is the primary language used. A simulator has been designed for the system, with the purpose of simulating the complete functionality of a bus. This has been created as a WPF application, with the .NET framework, and with C\# as a language.\\
For the purpose of project management, has parts of scrum been used, and through this the V-model was found to be very useful.\\
The final product is a system, where the user can be kept updates in regard the the position of a bus, and  where an administrator easily can perform management.


\end{document}