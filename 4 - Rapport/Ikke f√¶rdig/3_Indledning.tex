\documentclass[Main.tex]{subfiles} 
\begin{document}

\section{Indledning}
En bus f�lger en ruteplan, men det er ikke altid at bussen er ved et givent stoppested, pr�cis p� det tidspunkt det forekommer i ruteplanen. Det vil derfor v�re gavnligt at kunne vide, pr�cis hvor en bus er, og hvor lang tid der er, til den n�rmeste er ved et givet stoppested. Denne viden vil, for det f�rste, give brugeren en st�rre chance for at n� sin bus og, for det andet, med sikkerhed vide, om en bus er k�rt fra et givet stoppested. Dette dokument beskriver udviklingen af en mobilapplikation der kan vise rute, stoppesteder og busser p� en rute, samt vise tiden til et stoppested for en bus. Desuden beskriver dokumentet ogs� oprettelsen af et administrator v�rkt�j, hvori busser, ruter og stoppesteder kan vedligeholdes. Mobilapplikationen er designet til android, men kan nemt genskabes p� en anden platform, da samtlige funktionaliteter ligger p� en server. Serveren er dog ikke klargjort til et distribueret system, men kan nemt skiftes ud med et der er, hvis det skulle v�re n�dvendigt.\\

\noindent
Kravene er blevet udarbejdet iterativt, da projeketet ikke har ekstern kunde, og s�ledes ikke repr�senterer kundens krav til systemet, men derimod udviklernes. En kunde er blevet simuleret, i form af en vejleder, s�ledes at alle krav-�ndringer og accepttest udf�rsel, blev gemmemg�et med hj�lp fra en ekstern kilde.\\

\noindent
TrackABus mobilapplikation og administrator hjemmeside, er blevet udhviklet til at h�ndtere system kravene. Systemet fungerer s�ledes, at en bruger kan v�lge en rute, s�dan samtligt persisteret information omkring ruten kan blive vist. Dette inkluderer busser, stoppesteder og, selvf�lgelig ruten. Brugeren kan herefter tilg� tids-funktionalitet ved at trykke p� et stoppested, hvorefter tiden til ankomst for den n�rmeste bus i begge retninger, vil blive vist. De komponenter brugeren kan tilg�, skal f�rst oprettes igennem administrations hjemmesiden. Denne del af systemet er defor den eneste, der kan �ndre distribuerede rute komponenter. I sammenh�ng med persitering bruges der to relationelle databaser; En distribueret, og en lokal til hver mobilapplikation. Den distribuerede h�ndtere samtlige information om de forskellige komponenter, hvor den lokale bruges i sammenh�ng med favorisering af ruter. Mobilapplikationen er baseret p� et eksamensprojekt i ITSMAP, lavet af gruppens to medlemmer, og betegnes derfor som legacy code. Hele systemet kan derfor ses som en videreudvikling af dette eksamensprojekt. Der eksisterer ingen yderligere krav, en dem projektgruppen selv har fastsat.\\

\noindent
Det har ikke v�ret muligt at tilg� reelle data for busser og ruter. Derfor har det v�ret n�dvendigt at bruge en del af arbejdsresourcerne p�, at designe og implementere et system til at kunne at kunne h�ndtere  disse data. Dette blev i sidste ende, til administrator hjemmesiden og simulatoren. \\
\noindent
I arbejdsprocess �jemed, er der blevet holdt daglige m�der i gruppen, hvor dagens arbejde blev diskuteret, samt det ugentlige m�l revuderet. Desuden blev der holdt ugentlige m�der med ad-hoc kunden, hvori projektes fremskriden blev forklaret og diskuteret.




\end{document}