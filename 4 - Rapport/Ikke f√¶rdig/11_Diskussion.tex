\documentclass[Main.tex]{subfiles} 
\begin{document}

\section{Diskussion af opn�ede resultater}

Som det kan ses i overst�ende afsnit, lykkes det at udvikle et software system, der kan vise en pr�cis tid til n�sten bus ankommer til et valgt stoppested, hvilket er hoved funktionaliteten i hele systemet. Ydermere er det muligt at nemt oprette nye busruter, stoppesteder og busser og persistere dem p� en database, s� det nemt kan blive vist p� mobil applikationen.\\
\\
Selve mobil applikationen k�re stabilt, der er ogs� implementeret fejlh�ndtering s�ledes, at skulle der tabes forbindelse til internettet vil dette blive h�ndteret samt brugeren vil f� besked om dette. Bussens placering og tiden til den ankommer til stoppested er begge meget pr�cise og ligger vel indefor et acceptable niveau. Brugergr�nsefladen der er blevet udviklet til mobil applikationen, er blevet gjort brugervenlig og overskuelig, knapperne er tydlige og med beskrivende tekst. Information bliver vist, s� der ikke er tvivl om hvad der bliver vist.\\
\\
F�r det var muligt at f� indtegnet ruter, stoppesteder og busser p� mobil telefonen, skulle der v�re en nem m�de at oprette disse, og tilf�je dem til databasen. Til dette form�l blev administrations hjemmesiden udviklet. Det er muligt herfra, nemt og simple, at oprette nye busruter, stoppesteder og busser, enda tilf�je busser til busruter. Brugergr�nsefladen kunne dog godt forbedres, og gjort mere simple, dette kan dog nemt g�res da der er blevet brugt 3 lags model i form af MVC.\\
\\
Den relationelle database indeholde alt det persisterede data systemet skal bruge, samt med den bliver brugt til at udregne tiden til en bus ankommer til et stoppested ved brug af en stored procedure, for at fjerne udregnigner og arbejde fra telefonen. Telefonen gemmer dog ogs� information lokalt, dette bliver gjort af to grunde, f�rst for at g�re det nemmere for brugeren at tilg� favorit ruter, og for at mindske telefonens dataforbrug.\\
\\
Web servicen er blevet lavet for at fjerne alt det tunge arbejde fra telefonen, og overf�re det til en hurtigere server, desuden er den blevet udviklet for at telefonen ikke tilg�r databasen direkte, da dette ville skabe et stort sikkerhedsproblem.\\
\\
Da det ikke var muligt at f� adgang til GPS-position for rigtige busser, blev der udviklet en bus-simulator. En nem og simple l�sning blev udviklet, s� det hurtigt var muligt at simulere en busser der k�re p� busruter.
\end{document}