\documentclass[Main.tex]{subfiles} 
\begin{document}

\section{Projektgennemf�rsel}
%Hvordan er projektet gennemf�rt mht. til tidsplan og iterationer, dvs. hvor mange iterationer vi har haft, deres varighed, samt hvordan det har v�ret med til at hj�lpe med projektstyringen. Det er vigtigt at arbejdsprocesser IKKE n�vnes i dette afsnit!

Gennem projektet er der fokuseret p� mange og korte iterationer. 
Dette har resulteret i 6 iterationer med en varighed p� 14-20 dage, hvilket har hjulpet med til at dele projektet op i mindre dele. 
\\
Desuden var �nsket, at hver iteration skulle resultere i en implementeret funktionalitet f.eks. en IDE eller muligheden for at sortere en enkelt klods.
Fordelen ved denne fremgangsm�de er, at der kunne planl�gges p� erfaringer og feedback fra tidligere iterationer samtidig med, at det ville give en forsikring om, at projektet ville blive f�rdigt til tiden.
\\
I praksis var dette dog sv�rt at overholde, da der af og til opstod problemer, som forhindrede at funktionaliteten var f�rdig ved fuldendt sprint. I nogle sprint var dokumenteringen af projektet sat i h�js�det, s� derfor var der ikke decideret moment for at fremvise funktionelt kode.
\\
Nu skal det ikke hede sig, at der slet ikke blev afholdt nogle af Scrum's sprint reviews. Det b�r n�vnes, at der blandt andet blev holdt en demonstration af b.la. v�gt-komponenten for kunden, da denne var f�rdig og funktionelt klar. Desuden er der l�bende blevet vist k�rende software, samt pr�senteret de f�rdige resultater omkring komponenten fra sprintet. Tilmed kan det n�vnes at sm� delm�l i sorteringen af en klods er blevet forelagt for kunden.

\subfile{5_1_Sprint1}
\subfile{5_2_Sprint2}
\subfile{5_3_Sprint3}
\subfile{5_4_Sprint4}
\subfile{5_5_Sprint5}
\subfile{5_6_Sprint6}


\end{document}