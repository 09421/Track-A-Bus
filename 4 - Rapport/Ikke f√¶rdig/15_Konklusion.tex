\documentclass[Main.tex]{subfiles} 
\begin{document}

\section{Konklusion}
%Konklusion p� alle de andre afsnit, dog prim�rt indledning, resultater og  diskussion af opn�ede. Derfor skal personerne der laver dette afsnit have l�st de andre afsnit grundigt, da den skal konkludere b�de p� processen og p� selve produktet. Negative sider m� ogs� gerne fremh�ves, men igen: Lad v�re med at s�lge os selv! Det er de store linjer der konkluderes p�, da resultaterne og processen allerede er diskuteret i tidligere afsnit.

TrackABus systemet er produktet af en id�, der blev realiseret, igennem en velfungerende arbejdsprocess og godt brug af udviklingsmetoder, dette bliver tydeligt afspejlet i det endelige produkt. De arbejdsprocesser der er blevet brugt, har v�ret stor hj�lp til at g�re, et ellers kompleks systemet, overskueligt. I starten blev der brug meget tid, p� at udforme krav, funktionaliteter og muligheder for hvordan opgaverne bedst kunne blive l�st. Dette skabte et solidt fundament til at opbygge hele systemet fra. Alle delene af systemet er fra starten blevet opbygget efter en lagdeling, for at g�re koden overskuelig og nem at udskifte dele, da dette har v�ret en iterativ udviklingsprocess, har det ofte v�ret n�dvendig at udskifte store dele af systemet, efter ny viden blev opn�ede.\\
\\
Det kan konkluderes at, det endelige systemet lever op til alle forventninger, og opfylder alle de krav og �nsker der i starten blev omstillet, som systemet skulle kunne udf�re. Systemet er brugervendligt, og det er yderst simple at f� vist sin �nskede busrute, og finde tiden til den n�ste bus ankommer til et stoppested p� mobil applikationen. Desuden er det simplet for administratoren at tilf�je nye busruter, stoppesteder og busser til systemet, og flytter busser mellem de forskellige busruter.\\
\\
Simulatoren der blev udviklet, da data fra virkelige busser ikke var tilg�ngelige, har v�ret til yderst stor hj�lp n�r forskellige dele af systemet skulle testes og fremvises. Gjorde det simple og nemt at teste forskellige dele af systemet, s� som algoritmen der udregner tid, og om information blev opdateret korrekt p� sk�rmen n�r bussen passerede det valgte stoppested. Uden at skulle v�re afh�ngig af, hvordan og hvor en virkelig bus befandt sig.\\
\\
Hvorom alting er, kan det konkluderes at en god ide, sat sammen med en velfungerende arbejdsprocess, gode udviklingsmetoder og fantastisk arbejdsmorale har medf�rt, at der er endt op med et s�rdeles velfungerende system. 
\end{document}