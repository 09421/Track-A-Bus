\documentclass[Main.tex]{subfiles} 
\begin{document}

\section{Konklusion}
%Konklusion p� alle de andre afsnit, dog prim�rt indledning, resultater og  diskussion af opn�ede. Derfor skal personerne der laver dette afsnit have l�st de andre afsnit grundigt, da den skal konkludere b�de p� processen og p� selve produktet. Negative sider m� ogs� gerne fremh�ves, men igen: Lad v�re med at s�lge os selv! Det er de store linjer der konkluderes p�, da resultaterne og processen allerede er diskuteret i tidligere afsnit.

TrackABus systemet er produktet af en id�, der blev realiseret, igennem en velfungerende arbejdsprocess og godt brug af udviklingsmetoder, dette bliver tydeligt afspejlet i det endelige produkt. De arbejdsprocesser der er blevet brugt, har v�ret stor hj�lp til at g�re, et ellers kompleks systemet, overskueligt. I starten blev der brug meget tid, p� at udforme krav, funktionaliteter og muligheder for hvordan opgaverne bedst kunne blive l�st. Dette skabte et solidt fundament at opbygge hele systemet fra. Alle delene af systemet er fra starten blevet opbygget efter en lagdeling, for at g�re koden overskuelig og nem at udskifte dele, da dette har v�ret en iterativ udviklingsprocess, har det ofte v�ret n�dvendig at udskifte store dele af systemet, efter ny viden opn�ede.
\end{document}