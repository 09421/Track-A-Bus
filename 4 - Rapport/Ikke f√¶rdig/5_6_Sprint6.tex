\documentclass[Main.tex]{subfiles} 
\begin{document}

\subsection{Sprint 5}
Sprint 5, som forl�b i ugerne 46 og 47, var anden del contruction fasen.\\
Med et fungerende system f�rdiggjort i sprint 5, blev fokuset lagt p� f�rdigg�relse af de sidste funktionaliter. Dette indebar oprettelse af komplekse rute p� kryds af alle komponenterne. Dette kr�vede en vis m�ngde refaktorering. Det indebar at lave database designet, s�vel som procedurene og funktionerne herp�, til at underst�tte denne funktionalitet. Hjemmesidens ruteadministrerings v�rkt�j, blev herunder ogs� videreudviklet, s� disse typer ruter kunne oprettes. Herudover blev en ny brugergr�nseflade udviklet, og funktionerne hertil blev opdateret. Udvidelsen af simulatoren kr�vede ikke mange �ndringer, da funktionalitet for at v�lge en rute, ikke var f�rdiggjort, og busser derfor altid k�rte p� den samme rute. Dog blev database tilgangen for denne lavet fuldst�ndigt om, da der var foretaget v�sentlige �ndringer p� den distribuerede database.\\
De fleste af mobil applikationens funktionaliteter blev ogs� udviklet i dette sprint. Dette blev gjort da denne var baseret p� legacy code, og de fleste implementeringer var udviklet p� forh�nd. Det var dog n�dvendigt at refaktorere store dele af denne, da det ikke levede op til de fastsatte design- og kodestandarder. 

\end{document}