\documentclass[Main.tex]{subfiles} 
\begin{document}

\section{Konklusion}
%Konklusion p� alle de andre afsnit, dog prim�rt indledning, resultater og  diskussion af opn�ede. Derfor skal personerne der laver dette afsnit have l�st de andre afsnit grundigt, da den skal konkludere b�de p� processen og p� selve produktet. Negative sider m� ogs� gerne fremh�ves, men igen: Lad v�re med at s�lge os selv! Det er de store linjer der konkluderes p�, da resultaterne og processen allerede er diskuteret i tidligere afsnit.
Silver Bullet Sort programmet er resultatet af en velfungerende arbejdsproces, og dette kan ogs� ses p� det endelige produkt. De agile udviklingsprocesser og rammer har v�ret en stor hj�lp til udviklingen af softwaren, og de har v�ret med til at give et overblik over arbejdsgangen. At der i starten af projektet blev brugt tid p� at fastl�gge kravene og unders�ge implementeringsmulighederne resulterede i, at der blev dannet et godt fundament for udviklingen af det samlede system. Hertil kan der knyttes denne kommentar at der fortl�bende er blevet tilf�jet funktionalitet til selve produktet, som ikke er blevet defineret som krav.\footnote{Se afsnit 14: Forslag til forbedringer af projektet eller produktet.}
Systemet er opbygget efter en lagdeling, s�ledes at koden bliver overskuelig og let kan videreudvikles, samtidig med at det er muligt at udskifte de forskellige lag. Denne lagdeling har ogs� resulteret i, at det var let at arbejde forholdsvist uafh�ngigt af hinanden, hvilket var en stor fordel.
\\\\
Det kan konkluderes, at det endelige produkt lever op til de opstillede krav, hvilket kan ses i afsnittene \textit{Resultater} og \textit{Diskussion af opn�ede resultater}. Programmet er brugervenligt, og det er nemt at f� robotten til at sortere en klods efter materialetype vha. det indbyggede standardprogram. Desuden er det overskueligt at f�lge med i selve sorteringsprocessen, og informationer om denne process kan til enhver tid tilg�s, da de som n�vnt er persisteret p� en database. Det kan yderligere konkluderes, at IDE'en opfylder sit m�l idet, at det er muligt at lave brugerdefinerede programmer. Disse programmer kan bruges p� den fysiske robot, og hvis denne ikke er til r�dighed, kan de simuleres, s�ledes at informationer om programmets k�rsel bliver vist. Heraf kan det konkluderes, at simuleringen ogs� virker efter hensigten. 
\\\\
Hvorom alting er, kan det konkluderes at de mange m�der med virksomheden, har medf�rt at det endelig produkt stemmer overens med virksomhedens �nsker og forventninger. \\
\end{document}