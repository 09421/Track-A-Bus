\documentclass[Main.tex]{subfiles} 
\begin{document}

\section{Projektets fortr�ffeligheder}
Nedenfor er en r�kke funktionaliteter og arkitekturer i projektet, som gruppen er specielt stolt af, beskrevet.
\\\\
\textbf{Mobil applikation}\\
Da mobil applikationen var den vigtigste del af projektet, blev der sat stort fokus p�, at f� bygget denne med et godt design. Der blev ogs� sat fokus p� brugervenlighed samt intuitivitet. Dette blev opn�et ved, at brugeren skal fortage s� f� interaktioner med mobil applikation som muligt, for at blive vist �nsket information. Samtidig er det tydeligt, hvad de forskellige knapper bruges til.\\
\\
\textbf{Web service}\\
Der blev hurtigt fastsl�et, at der skulle ske s� lidt arbejde i mobil applikation som muligt. Dette blev l�st ved at abstrahere tilgangen til den distribuerede database og alle dennes funktioner, v�k fra mobil applikationen og ind p� en web service. Dette har gjort det muligt at udf�re meget dataprocessering p� en server, i stedet for en telefonen. Desuden muligg�res det, at nemmere kunne udvikle lignende applikationer til andre mobile styresystemet s� som iOS og Windows Phone.
\\\\
\textbf{Ruteindtegning}\\
Ruteindtegnings delen p� administrations hjemmesiden s�rger for at der intuitivt kan oprettes nye busruter. Dette sker igennem meget f� klik p� et kort, og derfor meget lidt administrator interaktion. Efter der er trykket p� "Save route"-knappen foreg�r samtlige punkt udtr�k af ruten automatisk, samtidig med at stoppestedernes position p� ruten selv udregnes. 
\end{document}