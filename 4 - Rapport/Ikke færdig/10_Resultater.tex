\documentclass[Main.tex]{subfiles} 
\begin{document}

\section{Resultater}
%OBJEKTIV vurdering af resultater. Meget gerne i tabelform, grafer og billeder. Der skal ikke v�re nogen form for diskussion i dette afsnit. Dette kommer i n�ste afsnit. 
Nedenfor er de mest v�sentlige resultater listet, hvorefter de kort beskrives. Der refereres til accepttestspecifikationen, hvis det �nskes at f� et st�rre overblik over de udf�rte tests.
\\\
\textbf{Overordnede resultater}
\begin{itemize}
\item F� en pr�cis tid til bus ankommer til stoppested.
\item En m�de at administrere busruter.
\item relationel database, til persistering af data.
\item Web service. til database adgang for mobil applikationen.
\item Simulator af bus.
\end{itemize}
\textbf{Mobil applikation}\\
Mobil applikationen kan vise busruter, med sens stoppesteder og busser der er p� ruten. samt det er muligt at f� vist tiden til den n�ste bus ankommer ved et valgt stoppested.\\
\\
\textbf{Administrations v�rkt�j}\\
Det kan lade sig g�re, at administrer busruter, dette indeb�rer at lave, slette og �ndre i ruter, stoppesteder og busser, samt �ndre hvilken rute de forskellige busser k�re op.\\
\\
\textbf{Database}\\
Databasen er skabt med MySQL og best�r af 8 tabeller til at holde alt data for systemet. Systemet gemmer en del data, herunder busruter, stoppesteder, busser og GPS-koordinater for de forskellige busser.\\
\\
\textbf{Web service}\\
Web servicen bruges som mellemled mellem mobil applikationen og MySQL databasen, samt fjerner meget af det tunge arbejde fra mobil applikationen og flytter det over p� en server.\\
\\
\textbf{Simulator}\\
Det er muligt at simulere en eller flere busser der k�re p� sin busrute.\\
\end{document}