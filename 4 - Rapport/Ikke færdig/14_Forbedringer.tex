\documentclass[Main.tex]{subfiles} 
\begin{document}

\section{Forslag til forbedringer af projektet eller produktet}
%Forslag til forbedringer i projektet. Lad v�re med at v�re for kritisk over for os selv, og s�lge os selv!!!!!! Forklar ogs� gerne hvorfor forbedringerne ikke blev implementeret. 
Nedenfor beskrives en r�kke elementer i projektet, som gerne s�s optimeret, hvis tiden og mulighederne tillod det.
\\\\
\textbf{GUI - Administrations hjemmeside}\\
Brugergr�nsefladen p� administrations hjemmesiden kunne v�re bedre. Hvis vinduet p� hjemmesiden mindskes vil de forskellige elementer ikke l�ngere passe i deres respektive felter. Dette kan nemt blive l�st ved at lave et nyt stylesheet, der vil g�re hjemmesiden til en mere behagelig brugeroplevelse. Der er dog blevet fokuseret mere p� funktionalitet frem for smuksering i dette projekt og der er derfor ikke blevet sat tid af til dette.\\
\\
\textbf{Server}\\
Serveren der hoster b�de administrations hjemmesiden, web servicen og MySQL databasen, er en bilig, men ikke s�rlig kraftfuld, server og kan derfor sandsynligvis ikke h�ndtere et stort antal samtidige brugere af mobil appliaktionen. Dette er dog ikke n�dvendigt i udviklings �jemed, men i tif�lde af at systemet skal distribueres, kan hele funktionaliteten p� serven nemt flyttes til en anden, mere kraftig, server.
\\
\textbf{Administrator login}\\
Administrator rettigheder skal kunne godkendes, f�r hjemmesiden kan tilg�s. Dette er ikke blevet implementeret, da det blev set som en bebyrdelse i udviklingen af systemet og dette ville betyde et login, ved hver test. I tilf�lde af at systemet skal distribueres, skal denne funktionalitet dog implementeres.
\\

\end{document}