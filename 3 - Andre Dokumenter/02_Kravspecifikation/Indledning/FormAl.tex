\section{Indledning}

\subsection{Form�l}


Dette dokument har til form�l at opstille de krav, der skal v�re opfyldt, n�r projektet er f�rdiggjort. Dokumentet er blevet udformet af bachelorgruppe 13038, best�ende af Christoffer Lousdahl Werge (studienr. 10832) og Lasse Lindsted S�rensen (studienr. 09421). Disse personer st�r til ansvar for, kravene sat i dette dokument er implementeret ved aflevering. Der er ikke nogen kunde for dette projekt, men vejleder Michael Alr�e agerer kunde i et kvalitetskontrol �jemed. �ndringer i dokumentet efter diskuteres derfor med ham, f�lgende af �ndringerne forklares.   

Kravene er opstillet som en r�kke Use Cases, og de funktionelle krav er derved omdrejningspunktet for dette dokument. De opstillede Use Cases beskriver, hvordan brugeren interragerer med systemet, i en ikke-implementerings specifik forstand.
\\
Det er essentielt at kravene i dette dokument skal f�lges, og ved en endelig accepttest skal alle Use Cases v�re opfyldt til kundens accept.

\subsection{Referencer}
F�lgende dokumenter refereres der til i kravspecifikationen
\begin{itemize}
	\item SBS\_Ide-funktioner.
	\item SBS\_HardwareSpec.
	\item Scorbot-ER 4u, Users Manual -- 100343b ER 4u.
\end{itemize}\
Udover disse dokumenter, er der lavet en accepttest specifikation, der st�r for at verificere at kravene er opfyldt.

\subsection{L�sevejledning}
Det forl�bige navn for projekt er TrackABus, n�r dette navn n�vnes i dokumentet, hentydes der til selve produktet. Der ses herunder en kort beskrivelse af de forskellige afsnit:
\\
\begin{itemize}
\item General Beskrivelse\\
I starten af dokumentet beskrives og illustreres systemet i sin helhed, samt gives der et overblik over systemet funktioner.\\

\item Funktionelle krav\\
Her er alle Use Cases beskrevet fully dressed.\\

\item Eksterne Gr�nseflader\\
Her ses en skitse over hvordan brugergr�nsefladen vil komme til at se ud.

\item Kvalitetsfaktorer og Designkrav\\
Her er beskrevet den kvalitet og den ydelse kunden kan forvente. Derudover kan her ogs� l�ses, hvilke krav der stilles til designer, i forbindelse med implementering\\


\end{itemize}
