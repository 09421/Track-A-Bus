\subsection{Use Case 5: Test program}
\textbf{M�l:}
\\
M�let med denne Use Case er at teste er et f�rdigt program eller en programsekvens via en simulering af robotten.
Det giver programm�ren mulighed for at teste sit program uden brug af robotten.
\\
\\
\textbf{Initiering:}
\\
Programm�ren starter tilkendegiver overfor systemet, at han vil starte en simulering.
\\
\\
\textbf{Akt�rer og interessenter:}
\\
Prim�re akt�rer:
\begin{itemize}
\item Programm�ren.
\end{itemize}
Sekund�re akt�rer:
\begin{itemize}
\item Ingen.
\end{itemize}
\textbf{Antal samtidige forekomster:}
\\
En samtidig forekomst.
\\
\\
\textbf{Ikke funktionelle krav:}
\begin{itemize}
\item Simuleringsbeskeder udskrives med et tidspunkt for begivenheden sammen med informationer om selve handlingen.
\end{itemize}
\textbf{Referencer:}
\\
Ingen.
\\\\
\textbf{Startbetingelser:}
\\
Det foruds�ttes, at brugeren er logget ind som programm�r.
\\
\\
\textbf{Slutresultat ved succes:}
\\
Programm�ren f�r fuldendt en testsimulering af programmet.
\\
\\
\textbf{Slutresultat ved undtagelser:}
\\
Programm�ren f�r ikke fuldendt en testsimulering af programmet
\\
\\
\textbf{Normalforl�b:}
\begin{enumerate}
\item Programm�ren tilkendegiver over for systemet at vedkommende �nsker en simulering.
\item Systemet indl�ser tilg�ngelige programmer.
\item Programm�ren v�lger en af programmerne.
\item Programm�ren starter simuleringen af programmet.
\item Alle systemets informationer fra programmet gemmes i en simuleringslogfil.
\item Simulationen afsluttes.
\end{enumerate}

\textbf{Undtagelser:}
\begin{enumerate}[\text{Undtagelse *} a:] 
\item Programm�ren trykker p� stop.
	\begin{enumerate}[1.]
	\item Simuleringen stoppes.
	\end{enumerate}
\end{enumerate}


