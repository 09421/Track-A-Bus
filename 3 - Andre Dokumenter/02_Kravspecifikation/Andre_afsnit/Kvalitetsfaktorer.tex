\section{Kvalitetsfaktorer}
Herunder er opstillet nogle kvalitetsfaktorer. Hver kvalitetsfaktorer har f�et en v�rdi ud fra f�lgende skala:\\
\textbf{1: Ubetydelig} \ \ \textbf{2: Ikke s�rlig vigtig} \ \ \textbf{3: Vigtig} \ \ \textbf{4: Meget vigtig} \ \ \textbf{5: S�rdeles vigtig}
\\
\begin{itemize}
\item \textbf{P�lidelighed: 5} \\
Det vigtigste er at systemet, virker og udf�rer det den er sat til. Det skal ske p� en p�lidelig m�de, hvor man kan regne med at robotten udf�rer det den er sat til, og giver besked, hvis fejl skulle v�re opst�et.
\item \textbf{Effektivitet: 2} \\
P�lideligheden er sat over effektiviteten, og hurtighed og lignende ses derfor ikke som s�rlig vigtig.
\item \textbf{Udvidelsesvenlighed: 4} \\
Udvidelsesvenlighed ses ogs� som meget vigtigt, da robotten skal kunne omprogrammeres til at virke i forskellige omgivelser.
\item \textbf{Brugervenlighed: 3} \\
Dette er en vigtig faktorer, men ikke altafg�rende for om systemet vil fungerer, da det ogs� kommer an p�, hvor godt operat�ren og program�ren bliver sat ind i systemet
\item \textbf{Vedligeholdelse og genbrugbarhed: 5} \\
Dette er helt klart noget der vil bliver bestr�bt, da det disse bliver opn�et via en god objektorienteret programmeringsstil.  
\end{itemize}