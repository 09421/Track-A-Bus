\documentclass[main]{subfiles}
\begin{document}
\subsubsection{Use Case scenarier}
Denne Use Case har 3 normalforl�b, idet det b�de er muligt at tilf�je ny busrute til systemet, fjerne en busrute fra systemet, samt �ndre i en busrute der findes i systemet. Det er kun en administrator der kan initialisere denne Use Case.
Normalforl�b 1 beskriver hvorledes der kan tilf�jes en ny busrute til systemet. Dette forg�r ved at administratoren tilkendegiver overfor systemet at han �nsker at oprette en ny busrute. Systemet pr�sentere nu administratoren for et kort. Administratoren kan nu indtegne en busrute p� dette kort. N�r den �nskede busrute er indtegnet p� kortet, kan busruten gemmes p� databasen, ved at brugeren tilkendegiver overfor systemet at busruten �nskes gemmes.

Normalforl�b 2 beskriver hvordan administratoren kan �ndre i en allerede eksiterende busrute.
Dette forg�r ved at administratoren v�lger en busrute, fra listen over busruter der findes i systemet. Administratoren tilkendegiver nu overfor systemet at han �nsker at �ndre i den valgte busrute. Systemet pr�sentere nu brugeren for et kort, med indtegnet busrute. Administratoren kan nu �ndre busrute som �nskes. �nskes �ndringerne at gemmes, kan administratoren tilkendegive overfor systemetat dette �nskes, hvorp� systemet vil gemme �ndringerne i databasen.

Normalforl�b 3 beskriver hvordan administratoren kan fjerne en allerede eksiterende busrute fra systemet.
Dette forg�r ved at administratoren v�lger en busrute, fra listen over busruter der findes i systemet. Administratoren tilkendegiver nu overfor systemet at den valgte busrute �nskes slettes fra systemet. Systemet sletter busruten fra databasen.

\subsubsection{Use Case undtagelser}
Da ruten b�de skal gemmes p� en database, samt hentes fra en database, er der risiko for at forbindelsen til databasen mistes. Hvis dette sker, vil brugeren blive pr�senteres for en besked om at det ikke er muligt at etablere forbindelset til databasen.
Hvis administratoren �nsker at annullere processen efter at have fortaget �ndringer vil administratoren blive pr�senteret for en besked, der sp�rger om der �nskes at stoppe uden at gemme. hvis administratoren v�lger at stoppe uden at gemme, retuneres til startsk�rmen. 

\end{document}