\documentclass[Main.tex]{subfiles} 
\begin{document}

\newpage

\subsubsection{Komponent 4: Mobile service}
Denne komponent har til form�l at v�re mellemled mellem mobil applikationen og MySQL databasen. komponenten er blevet lavet, da mobil applikationen ikke m� have direkte adgang til en databasen, da dette har store sikkerhedsmessige implikationer. Uden denne service vil det ogs� v�re muligt for ondsindet brugere at tilg� databasen og manipulere med data p� en ikke �nsket m�de. Et andet form�l med denne web service er at g�re det g�re det nemt at udvikle ny mobil applikation, til et hvilkensomhelst styresystem, uden at skulle t�nkte p� database tilgang.\\\\
\textbf{Design:}\\
Mobil servicen bliver brugt til at hente data fra MySQL databasen som mobil applikationen skal bruge. Dette indeb�re at hente en liste af busruter, hente en bestemt rute, hente stoppestederne for en rute, hente positionen for alle busser p� ruten og kalde den Stored procedure der udregner tiden f�r der er en bus ved et valgt stoppested. Web servicen er tilg�ngelig for alle, da de eneste funktionaliteter den udbyder er at hente data fra databasen. Med en �ben web service er det muligt for alle at bruge data til at udvikle nye applikationer. p� \url{http://trackabus.dk/AndroidToMySQLWebService.asmx} er det muligt at se tilg�ngelige funktioner.\\
Her kan ses et eksempel p� hvad det kr�ver at kalde funkionen GetBusPos fra web servicen, ved brug af SOAP. Det kan ses at den kr�ver et 'busNumber' i form af en string, som input parameter.

\begin{lstlisting}[caption=request til service function GetBusPos, language=XML]
POST /AndroidToMySQLWebService.asmx HTTP/1.1
Host: trackabus.dk
Content-Type: application/soap+xml; charset=utf-8
Content-Length: length
<?xml version="1.0" encoding="utf-8"?>
<soap12:Envelope xmlns:xsi="http://www.w3.org/2001/XMLSchema-instance" xmlns:xsd="http://www.w3.org/2001/XMLSchema" xmlns:soap12="http://www.w3.org/2003/05/soap-envelope">
  <soap12:Body>
    <GetbusPos xmlns="http://TrackABus.dk/Webservice/">
      <busNumber>string</busNumber>
    </GetbusPos>
  </soap12:Body>
</soap12:Envelope>}
\end{lstlisting}
Herunder ses det SOAP response man f�r tilbage efter at have lavet det overst�ende kald til servicen. Det kan ses at man f�r en liste af points tilbage, hvor hver point indeholder en Lat, en Lng og et ID som alle er af datatypen string.

\begin{lstlisting}[caption=response fra service function GetBusPos, language=XML]
HTTP/1.1 200 OK
Content-Type: application/soap+xml; charset=utf-8
Content-Length: length
<?xml version="1.0" encoding="utf-8"?>
<soap12:Envelope xmlns:xsi="http://www.w3.org/2001/XMLSchema-instance" xmlns:xsd="http://www.w3.org/2001/XMLSchema" xmlns:soap12="http://www.w3.org/2003/05/soap-envelope">
  <soap12:Body>
    <GetbusPosResponse xmlns="http://TrackABus.dk/Webservice/">
      <GetbusPosResult>
        <Point>
          <Lat>string</Lat>
          <Lng>string</Lng>
          <ID>string</ID>
        </Point>
        <Point>
          <Lat>string</Lat>
          <Lng>string</Lng>
          <ID>string</ID>
        </Point>
      </GetbusPosResult>
    </GetbusPosResponse>
  </soap12:Body>
</soap12:Envelope>
\end{lstlisting}

\end{document}