 \documentclass[Main.tex]{subfiles} 
\begin{document}

\subsection{Implementering af persistens}
Persisteringen af data best�r af to hoveddele. En facade til den direkte tilgang og en besked k� der lavet efter observer design m�nsteret som bliver beskrevet i implementeringsviewet, afsnit 8.2.7-Observer Pattern
\\
For at tilg� databasen p� C\# siden er der blevet brugt facade designm�nsteret. Tilgangen til databasen sker alts� igennem �n klasse, s� vi bevarer lav kobling. Til selve forbindelsen blev der brugt SQLCommands. Dette er en del af .Net frameworket. Forbindelsen ligger hardcodet i facaden, dvs. at det ikke er op til brugeren af systemet, at definere hvilken database man tilg�r. Denne forbindelse best�r af en streng med bruger id, password, server, database og diverse optionelle valgmuligheder som f.eks. Connection Timeout som i dette tilf�lde bestemmer, hvor lang tid systemet maksimalt m� bruge p� at skabe forbindelse. P� kodeudsnit \ref{lst:SqlconnectionCode} kan man se hvordan forbindelsen bliver defineret og oprettet.
\begin{lstlisting}[caption=Definition and opening of connection to database, label={lst:SqlconnectionCode}]
private const string UserID     = "user id=F12I4PRJ4Gr5;";
private const string Password   = "password=F12I4PRJ4Gr5;";
private const string Server     = "server=webhotel10.iha.dk;"; 
private const string Database   = "database=F12I4PRJ4Gr5; ";
private const string ConnectionTimeOut = "Connection TimeOut=2";

private readonly SqlConnection _myConnection = new SqlConnection(
                                        UserID +
                                        Password +
                                        Server +
                                        Database +
                                        ConnectionTimeOut
                                        );
..............

if (_myConnection.State != ConnectionState.Open)
        _myConnection.Open();
\end{lstlisting}
Forbindelsen bliver brugt ved at eksekvere en SQL streng fra C\# siden, alts� en SQL statement skrevet i en string. Det er blevet fors�gt at g�re disse commands generelle, s� alt efter hvad man vil p� en vilk�rlig tabel, kan man tilg� en generel funktion, som specificeres via parametre. N�r en funktion, der henter noget fra databasen k�res, vil data returneres via en SQLDatareader som funktionen, ExecuteReader(), returnerer. Nedunder, p� kodeudsnit \ref{lst:SqlcommandCode} kan man se, hvordan en simpel SELECT statement bliver brugt, k�rt og hvordan den returnerer data.

\begin{lstlisting}[caption= Definition of a Select statement with a where clause, label={lst:SqlcommandCode}]
public List<string> SelectFrom(string columnName, 
						                   string tableName,
							   	             string whereColumnName,
							                 string target)
{		
      if (_myConnection.State != ConnectionState.Open)
                _myConnection.Open();
                
      var back = new List<string>();
      var command = new SqlCommand(string.Format(
      			          "SELECT {0} FROM {1} WHERE {2}='{3}'",
      		              columnName, tableName,
      		              whereColumnName, 
      		              target), 
      		              _myConnection);
      		              
      var read = command.ExecuteReader();
      while (read.Read())
      {
            back.Add(read[columnName].ToString().TrimEnd(' '));
                
      }
      read.Close();
      _myConnection.Close();
      return back;
}
\end{lstlisting}
\noindent
N�r man kalder Execute reader bliver den specificerede streng k�rt p� databasen. Hvis operationen p� databasen returnerer noget vil dette blive hentet via read[columnName] hvor columnName er den kolonne man vil hente fra. Hvis der hentes flere kolonne fra databasen vil disse blive gemt i den kolonne man specificerer. Database tilgangen sker p� samme m�de ved samtlige funktioner, med den undtagelse, at der ikke returneres noget p� de operationer p� databasen, der ikke skal returnere noget f.eks. insert- eller update statements. Disse operationer er de eneste der forekommer p� databasen, udover de triggers og functions der ogs� kan k�res.

\end{document}