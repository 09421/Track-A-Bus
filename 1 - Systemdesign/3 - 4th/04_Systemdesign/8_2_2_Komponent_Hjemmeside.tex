\documentclass[Main.tex]{subfiles} 
\begin{document}

\subsubsection{Komponent 3: Administrations hjemmeside }
Denne komponent har til form�l at h�ndtere alle de adminitrative opgaver i system. Dette best�r af 4 delkomponenter:

\begin{itemize}
\item Den f�rste delkomponent g�r det muligt at tilf�je en bus til systemet, fjerne den, eller rediger i en bus der allerede findes i systemet.
\item Derefter skal det v�re muligt at tilf�je eller fjerne en bus fra en rute der findes i systemet.
\item Den tredje delkomponent g�r det muligt at kunne oprette en hel ny busrute i systemet, �ndrer i en allerede eksiterende busrute, eller slette en fra systemet.
\item Den sisdte delkomponent best�r af muligheden for at kunne tilf�je, �ndre samt fjerne busstoppesteder fra systemet.
\end{itemize}

Alle disse delkomponenter udg�re tilsammen en vigtig del af systemet, da uden nogle af dem vil det ikke v�re muligt at kunne f� vist nogle af overst�ende ting p� mobil applikationen.\\\\

\textbf{Design:}\\

Hjemmesiden er blevet implementeret ved brug af Microsoft ASP.NET MVC 4 frameworket. Dette g�r det nemt og hurtigt at implementere en sofistikeret og moderne hjemmeside, der f�lger gode design principper. MVC st�r for Model-View-Controller og f�lger de samme principper som MVVM ang�ende 'separation of concerns'.\\
For at kunne indtegne busruter og stoppesteder skal der bruges et kort, til dette er der blevet brugt Google maps samt Google Directions API.  

Hjemmesiden best�r af 4 view,f�rst og fremmest et view til startsiden der linker til de 3 andre views, der best�r af et der h�ndtere alt vedr�rende busser, et til stoppesteder samt et til busruter.
Det f�rste view der h�ndtere alt om busserne

%\begin{lstlisting}[caption=Funktionen Initialization(...)]
%public bool Initialization(short sMode, short sSystemType)
%{
%    string sModeS;
%    string sSystemTypeS;
%    switch (sMode)
%    {
%        case 0:
%            sModeS = "INIT_MODE_DEFAULT selects last used mode (from ini file).";
%            break;
%        case 1:
%            sModeS = "INIT_MODE_ONLINE force online mode.";
%            break;
%        case 2:
%            sModeS = "INIT_MODE_SIMULAT selects simulation mode.";
%            break;
%        default:
%            sModeS = "Unknown INIT_MODE.";
%            break;
%    }
%    switch (sSystemType)
%    {
%        case 0:
%            sSystemTypeS = "DEFAULT_SYSTEM_TYPE let the libary detect the robot type.";
%            break;
%        case 41:
%            sSystemTypeS = "ER4USB_SYSTEM_TYPE define robot type as ER-4 Scorbot with USB connection.";
%            break;
%        default:
%            sSystemTypeS = "Unknown SYSTEM_TYPE.";
%            break;
%    }
%
%    logWindow.LogThis(loggingLevels.INFO, "Initialization: " + retVal, false);
%    logWindow.LogThis(loggingLevels.INFO, "Initialization MODE: " + sModeS + " ## " + sSystemTypeS, true);
%
%    logFile.LogThis(loggingLevels.INFO, "Initialization: " + retVal, false);
%    logFile.LogThis(loggingLevels.INFO, "Initialization MODE: " + sModeS + " ## " + sSystemTypeS, true);
%    logFile.LogThis(loggingLevels.INFO, "newline", false);
%
%    initRuned = true;
%    return retVal;
%}
%\end{lstlisting}



\end{document}