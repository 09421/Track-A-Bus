\documentclass[Main.tex]{subfiles} 
\begin{document}

\subsubsection{Use Case scenarier}
Det er b�de operat�ren og programm�ren, der kan tilg� denne Use Case. Den initialiseres ved, at en af akt�rerne tilkendegiver overfor systemet, han vil se gemte logfiler, hvorefter tilg�ngelige logfiler bliver listet. Herefter v�lger akt�ren hvilken logfil der skal vises, og s�ledes bliver data for denne logfil vist. Dern�st bliver logfilen gemt, ved at akt�ren tilkendegiver overfor systemet, at filen �nskes gemt. Akt�ren bliver dern�st pr�senteret for et vindue, hvor placering for logfilen kan v�lges, og hvor navnet p� filen kan indtastes. Til sidst gemmes logfilen det �nskede sted, med det �nskede navn. 

\subsubsection{Use Case Undtagelser}
Da databasen er sekund�r akt�r i Use Casen, og da logfiler hentes p� databasen, er der risiko for, at forbindelsen til databasen mistes. Hvis dette er tilf�ldet, bliver akt�ren spurgt, om han vil vente p� at forbindelsen bliver genetableret, eller om programmet skal lukke ned. V�lges der at vente, bliver han igen pr�senteret for samme besked, hvis forbindelsen ikke kunne genetableres. Hvis det er lykkedes at genetablere forbindelsen, k�rer systemet videre. Er der ingen gemte logfiler p� databasen, er det ikke muligt at f� vist data for en logfil.
\\
Akt�ren har mulighed for blot at f� vist en logfil, uden at f� denne gemt, og hvis dette er tilf�ldet, skal han blot lade v�re med at tilkendegive, at filen �nskes gemt. Hvis han fortryder at gemme filen, er det ogs� muligt at annullere handlingen. N�r navnet p� filen skal indtastes, er det muligt at akt�ren v�lger et navn, der allerede eksistere i den givne sti. Hvis dette er tilf�ldet, bliver han spurgt om han vil overskrive den eksisterende fil. Hvis han ikke vil dette, f�r han mulighed for at indtaste et andet navn, der ikke er optaget. 
\\


\end{document}