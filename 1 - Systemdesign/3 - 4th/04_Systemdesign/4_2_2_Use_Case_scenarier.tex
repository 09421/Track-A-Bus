\documentclass[Main.tex]{subfiles} 
\begin{document}

\subsubsection{Use Case scenarier}
Denne Use Case sortere klodser efter materialetype. 
Det kr�ves at robotten er i startposition, samt at den ikke har fat i en klods. 
Klodserne kommer sekventielt k�rerende p� transportb�ndet, og n�r en klods registreres af transportb�ndssensoren samles denne op af robotarmen. 
Her tilg�es Use \emph{Case 1.1: M�l og vej klods} (include). 
Klodsen vendes, s�ledes at alle sider m�les, og placeres slutteligt p� v�gten. 
Dern�st findes materialetypen gennem \emph{Use Case 1.2: Bestem matrialetype} (include). 
Her bliver densiteten udregnet, og materialetypen hentes i databasen, hvorefter den aktuelle klods data persisteres p� databasen.
Klodsen bliver dern�st placeret i sin respektive kasse, indeholdende den givne materialetype, hvorefter robotarmen k�rer til startposition.
\\
Et vigtigt ikke-funktionelt krav for denne Use Case er at det maksimalt m� tage 3 minutter at sorterer en klods, alts� fra klodsen er registeret p� transportb�ndet, til den succesfuldt er placeret i den korrekte kasse.

\subsubsection{Use Case undtagelser}
Undervejs i forl�bet har brugeren mulighed for at afvige fra normalforl�bet. 
Der kan trykkes p� enten en fysisk, eller softwarebaseret n�dstopknap, hvilket stopper sorteringen. Ligeledes kan sorteringsmekanismen stoppes ved tryk p� en pauseknap, som stopper sorteringsmekanismen midlertidigt, til det igen startes gennem brugergr�nsefladen.
\\
Da der undervejs i Use Casen skrives et loggingindl�g til databasen omkring, hvordan processen forl�ber, er der risiko for, at der mistes forbindelse til databasen. Hvis dette sker, gemmer systemet selv beskederne og persisterer dem p� databasen, n�r forbindelsen genetableres. Det samme sker, mht. klodsens data. Denne undtagelse er ogs� g�ldende for de to include Use Cases tilh�rende denne Use Case, da de ligeledes laver et logging indl�g til databasen. 
\\
Hvis der ikke er plads til klodsen i boksen med den tilh�rende materialetype, modtager operat�ren en besked herom, hvorefter robotarmen placerer klodsen til h�jre for sensoren p� transportb�ndet. Dette resulterer i, at klodsen kasseres, n�r sorteringsalgoritmen igen startes. 

\end{document}