\documentclass[Main.tex]{subfiles} 
\begin{document}

\subsection{K�rsels-hardware}

For at udnytte systemets fulde funktionalitet skal programmet v�re opkoblet til Scorebot-Er 4u og en v�gt. Dog skal det siges, at man kan simulere disse to elementer, s� man er dem foruden, selvf�lgelig uden at der kan sorteres klodser i det virkelige problemdom�ne.
Det er ydermere n�dvendigt at v�re forbundet til en database, hvor brugerdata ligger. Hvis det ikke er muligt kan der ikke logges ind p� systemet. Databasen kan godt simuleres efter login.

\subsubsection{Opstilling}
V�gtenprintet forbindes over med det 10-polede parallelkabel til STK500-kittets port A. 5-pin DIN stikket forbindes til v�gten. Mellem STK-kittets Spare-serialport og computerens USB forbindes prolific Seriel-to-usb-converter og denne indstilles som COM20. Desuden skal v�gtprint tilsluttes +5V, -5V og ground p� de respektive bananstik. \\
PCen skal have netv�rks forbindelse, og er der ikke tilsluttet til IHA's net skal der v�re forbindelse til skolens VPN. USB-stikket fra Scorebots USB-controller skal v�re tilsluttet en af PCens USB-porte.\\
Er transportb�ndets position ikke indstillet, skal transportb�ndet placeres i yderste venstre hj�rne af bordet.


\end{document}