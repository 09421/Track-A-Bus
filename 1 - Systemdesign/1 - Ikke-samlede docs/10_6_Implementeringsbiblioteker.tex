\documentclass[Main.tex]{subfiles} 
\begin{document}

\subsection{Implementeringsbiblioteker}
Da systemet er baseret p� .NET, og derfor er der selvf�lgelig brugt adskellige .NET biblioteker. Det er langt fra alle der vil blive beskrevet i dette afsnit, hvor kun de vigtigste er forklaret.

\begin{itemize}
\item System.IO.Ports - Bibliotek, som indeholder funtioner der bruges til kommunikation over den serielle port. Se klassen RobotProjekt.Weights.Weight.
\item WPF.Themes - Bibliotek der indeholder brugergr�nsefladens tema.
\item Microsoft.Scripting - Er blevet brugt til at lave en scripting-engine til brug i IDEen. Se komponentbeskrivelsen af IDEen under afsnit 8.
\item Galasoft.Mvvmlight - Bibliotek fra Galasoft, som bruges til MVVM funktiontionalitet. Fx indeholder det en implementering af RelayCommand, og et Message System.
\item NUnit.Framework - NUnit frameworket, som bruges til automatiserede test.
\end{itemize}


\end{document}