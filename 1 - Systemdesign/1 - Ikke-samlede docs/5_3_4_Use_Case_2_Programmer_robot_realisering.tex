\documentclass[Main.tex]{subfiles} 
\begin{document}

\subsubsection{Use Case 2: Programmer robot realisering}
Programmer robot Use Casen er blevet udf�rdiget ved at lave en IDE, som benytter ironpython samt microsoft.scripting biblioteket. Denne Use Case bliver initalizeret ved at brugeren trykker p� 'manage program' knappen fra menu vinduet. Herefter bliver brugeren taget til 'mange program' vinduet, hvor der er mulighed for at oprette, gemme, rediger, slette og indl�se programmer. Normalforl�b 1, 'Opret nyt program' bliver initialiseret n�r programm�ren begynder at skrive en ny robot sekvens i den richtextbox der er i vinduet, hvorefter programm�ren kan navngive og gemme det skrevet program. Normalforl�b 2, 'Rediger gammelt program' omhandler at kunne hente et gammel program, som programm�ren har skrevet, og gemt,  tidligere. Dette g�res ved at v�lge et program, fra listen over gemte programmer, og trykke p� 'Load' knappen. Herefter vil sekvensen for det valgte program blive vist i richtextbox'en. Normalforl�b 3, 'Slet program' omhandler at kunne slette et gemt program. Dette g�res ved at v�lge et gemt program fra listen, og derefter trykke p� 'Delete', dette vil slette det valgte program permanent. Normalforl�b 4, 'Anvend program' kr�ver at programm�ren, eller operat�ren, er i 'Run window' vinduet, her kan brugeren v�lge et af de gemte programmer fra en liste, hvorefter der trykkes p� 'Run' knappen(robotten skal v�re homet, f�r dette er muligt). Dette vil f� robotten til at udf�re sekvensen fra det valgte program.
Denne IDE er simpelt lavet, ved at g�re brug af ironpython, og microsoft.scripting biblioteket, som g�r det muligt at overs�tte python kode til C\# kode, som kan bruges til at styre robotten. Dette forg�r alt i IronPython klassen, som har en singleton scriptEngine med en Python engine. I det tilf�lge at brugeren �nsker at k�re et program p� robotten via. 'Run window', bliver funktionen Script(string, Dictionery<stirng, object>) kaldt, hvor den f�rste string parameter, er den python sekvens der st�r i den valgte fil, og Dictionery<stirng, object> parameteren som har et Dictionery med "Robot" og et object af klassen InterfaceRobotClass, som har alle de mulige robot realateret funktioner som brugeren kan kalde, via IDE'en. Ved dette kald bliver der lavet et script, dette script kan blive tjekket for syntaks fejl, ved at kalde SyntaxIsOK() som retunere en string med den f�rste fejl den finde, eller "No Syntax errors", hvis ingen er fundet.
SyntaxIsOk(), bliver ogs� kaldt, n�r brugeren trykker p� 'Compile' knappen, i 'programs window' vinduet.
funktionen Execute(), eksekvere det script, som script() generere, forudsat at der ikke er nogen fejl i det. denne eksekvering forg�r i sin egen tr�d, det g�r det muligt for brugeren at bruge de andre funktionaliteter i systems, imens robotten k�re.

intellisense er ogs� blevet implamenteret i den richtextbox som programm�ren skriver sin sekvens i, dette g�r det muligt at f� vist en liste over alle de mulige robot funktioner som er tilg�ngelige, for mere information om disse funktioner se \textit{SBS\_IDE-funktioner} documentet



\end{document}