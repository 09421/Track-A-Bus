\documentclass[Main.tex]{subfiles} 
\begin{document}
\subsubsection{Komponent 2: Mobile service}
Denne komponent har til form�l at v�re mellemled mellem mobil applikationen og MySQL databasen. komponenten er blevet lavet, da mobil applikationen ikke m� have direkte tilgang til MySQL databasen, da dette har store sikkerhedsm�ssige implikationer. Et andet form�l med denne web service er at g�re det g�re det nemt at udvikle til en anden mobil platform, da der ikke skal t�nkes p� database tilgang, men blot kaldes til servicen. Herudover abstraheres samtlige processeringstunge udregninger v�k fra applikationen.\\\\
\textbf{Design:}\\
Mobil servicen bliver brugt til at hente data fra MySQL databasen som mobil applikationen skal bruge. Dette indeb�re at hente en liste af busruter, hente en bestemt rute, hente stoppestederne for en rute, hente positionen for alle busser p� ruten og kalde den proceduren p� databasen som udregner tiden til ankomst for den n�rmeste bus ved et valgt stoppested. Tidsudregningen kan der l�ses mere om i afsnit \ref{9.1.3: Stored Procedures}. Web servicen er tilg�ngelig for alle, da de eneste funktionaliteter den udbyder er at hente data fra databasen. Med en �ben web service er det muligt for alle at bruge data til at udvikle nye applikationer. \footnote{For en liste af funktioner i servicen, se \url{http://trackabus.dk/AndroidToMySQLWebService.asmx} }
P� kodeudsnit \ref{lst:SoapStuff} kan der f�lges et eksempel p�, hvad det kr�ver at kalde funkionen GetBusPos fra web servicen, ved brug af SOAP version 1.1.\footnote{For mere information om SOAP, se \url{http://www.w3.org/TR/2000/NOTE-SOAP-20000508/} } Det kan ses at funktionen kr�ver en "busNumber" input parameter i form af en string. Implementeringen af SOAP p� mobil applikations siden kan findes under afsnit \textit{9.2.1: Implementering af persistens i mobil applikationen}. 

\begin{lstlisting}[caption=request til service function GetBusPos, language=XML, label={lst:SoapStuff}]
POST /AndroidToMySQLWebService.asmx HTTP/1.1
Host: trackabus.dk
Content-Type: application/soap+xml; charset=utf-8
Content-Length: length
<?xml version="1.0" encoding="utf-8"?>
<soap12:Envelope xmlns:xsi="http://www.w3.org/2001/XMLSchema-instance" xmlns:xsd="http://www.w3.org/2001/XMLSchema" xmlns:soap12="http://www.w3.org/2003/05/soap-envelope">
  <soap12:Body>
    <GetbusPos xmlns="http://TrackABus.dk/Webservice/">
      <busNumber>string</busNumber>
    </GetbusPos>
  </soap12:Body>
</soap12:Envelope>}
\end{lstlisting}

P� kodeudsnit \ref{lst:SoapResponse}, kan det SOAP response, servicen returner ved fuldent kald til GetBusPos. Svaret kan ses som et XML. SOAP implementeringen p� applikations siden, st�r for at h�ndtere l�sningen at dette tr�.  Det kan ses at der retuneres en liste af Points, hvor hver Point indeholder en Lat, en Lng og et ID som alle er af datatypen string.

\begin{lstlisting}[caption=response fra service function GetBusPos, language=XML, label={lst:SoapResponse}]
HTTP/1.1 200 OK
Content-Type: application/soap+xml; charset=utf-8
Content-Length: length
<?xml version="1.0" encoding="utf-8"?>
<soap12:Envelope xmlns:xsi="http://www.w3.org/2001/XMLSchema-instance" xmlns:xsd="http://www.w3.org/2001/XMLSchema" xmlns:soap12="http://www.w3.org/2003/05/soap-envelope">
  <soap12:Body>
    <GetbusPosResponse xmlns="http://TrackABus.dk/Webservice/">
      <GetbusPosResult>
        <Point>
          <Lat>string</Lat>
          <Lng>string</Lng>
          <ID>string</ID>
        </Point>
        <Point>
          <Lat>string</Lat>
          <Lng>string</Lng>
          <ID>string</ID>
        </Point>
      </GetbusPosResult>
    </GetbusPosResponse>
  </soap12:Body>
</soap12:Envelope>
\end{lstlisting}

\end{document}