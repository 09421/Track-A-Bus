\documentclass[main]{subfiles}
\begin{document}
\subsubsection{Use Case scenarier}
Denne Use Case har tre normalforl�b, idet det b�de er muligt at tilf�je en busrute, fjerne en busrute, samt �ndre i en busrute. Det er kun en administrator der kan initialisere denne Use Case.
Normalforl�b 1 beskriver hvordan en ny busrute kan tilf�jes. Dette foreg�r ved, at administratoren tilkendegiver overfor systemet at en ny busrute �nskes oprettet. Administratoren pr�senteres for et kort hvorp� det er muligt at indtegne en busrute. N�r den �nskede busrute er indtegnet p� kortet, kan den gemmes p� den distribuerede database ved, at administratoren tilkendegiver overfor systemet, at ruten �nskes gemt. \\

Normalforl�b 2 beskriver hvordan en allerede eksiterende busrute kan �ndres.
Dette foreg�r ved, at administratoren v�lger en busrute, fra listen af busruter der findes i systemet. Administratoren pr�senteres nu for et kort, hvorp� den valgte busrute er indtegnet. Nu kan den valgte busrute �ndres som det �nskes, hvorefter det kan tilkendegives, at �ndringerne �nskes gemmet. Systemet vil herefter gemme �ndringerne i den distribuerede database.\\

Normalforl�b 3 beskriver hvordan en allerede eksisterende busrute kan fjernes.
Dette forg�r ved at administratoren v�lger en busrute, fra listen af busruter der findes i systemet. Der tilkendegives nu overfor systemet, at den valgte rute �nskes slettet, hvorefter busruten slettes fra den distribuerede database.\\

\subsubsection{Use Case undtagelser}
Da informationen om ruten b�de skal hentes fra og gemmes p� en den distribuerede database, er der risiko for, at forbindelsen til databasen mistes. Hvis dette sker, vil administratoren blive pr�senteret for en besked om, at det ikke er muligt at etablere forbindelse til databasen.


\end{document}