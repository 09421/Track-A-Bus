\documentclass[Main.tex]{subfiles} 
\begin{document}

\subsection{Arkitektur m�l og begr�nsninger}

Der var ikke fastsat nogen funktionelle krav til udviklingsv�rt�jer eller arkitektur fra et produktopl�g, og derfor blev produktet udhviklet med de v�rkt�jer, der var mest erfaring i. Da projektet dog bygger ovenp� en eksamensopgave udf�rt i ITSMAP, som var Android baseret, blev det fastsat, at mobilapplikationen skulle udf�res p� samme Android platform.\footnote{For mere information, se ITSMAP\_TemaProjekt p� bilags CDen under Projekt Dokumenter} \\
Simulatoren er udbygget i WPF med .NET 4.5 frameworket, hvilket betyder at simulatoren skal bruges p� en Windows platform. \\
En dom�nenavn blev k�bt hos UnoEuro\footnote{\url{www.unoeuro.com}}, til at holde administrator delen af produktet, samt den distribuuerede database. Dette dom�ne satte visse begr�nsninger for systemet, da kun en MySQL database var supporteret, og der skulle v�lges imellem PHP og ASP.NET. Da ASP.NET er C\# baseret, blev denne mulighed valgt, da projektgruppen havde mere erfaring med dette, end med PHP.\\

\end{document}