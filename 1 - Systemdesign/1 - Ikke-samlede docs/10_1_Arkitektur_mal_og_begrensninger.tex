\documentclass[Main.tex]{subfiles} 
\begin{document}

\subsection{Arkitektur m�l og begr�nsninger}

Der var ikke fastsat nogen funktionelle krav til udviklingsv�rt�jer eller arkitektur fra produktopl�get, og derfor blev produktet udhviklet med de v�rkt�jer, der var mest erfaring i. Da projektet dog bygger ovenp� en eksamensopgave udf�rt i ITSMAP, som var android baseret, blev det fastsat, at mobilapplikationen skulle udf�res p� samme android platform. \\
Simulatoren er udbygget i WPF med .NET 4.5 frameworket, hvilket betyder at simulatoren skal bruges p� en Windows platform. \\
En dom�nenavn blev k�bt hos UnoEuro, til at holde administrator delen af produktet, samt den distruerede database. Dette dom�ne satte visse begr�nsninger for systemet, da kun en MySQL database var supporteret, og der skulle v�lges imellem PHP og ASP.NET. Da ASP.NET er C\# baseret, blev denne mulighed valgt, da projektgruppen havde mere erfaring med dette, end med PHP.\\


%Der er ikke mange funktionelle krav at indhente fra produktopl�get, og derfor har disse heller ikke haft den store indflydelse p� arkitekturen. Dog er systemet implementeret til brug p� windows pc, og brugergr�nsefladen er baseret p� WPF, og dette har en vis betydning for arkitekturen. Brugen af ScoreBot-robotarmen og USBC.dll biblioteket betyder ogs� at arkitekturen er udformet omkring disse ydre omst�ndigheder.
%For gruppens sider er der fastsat en r�kke udviklingskrav. F.eks. bruges Scrum som den prim�re udviklingsprocess og alle diagrammer er lavet i visual paradigm.

\end{document}