\documentclass[Main.tex]{subfiles} 
\begin{document}

\subsection{Dokumentstruktur og l�sevejledning}

\begin{itemize}
	\item[] \textbf{Afsnit 1 � INTRODUKTION:} Introducerer l�seren til projektet og dokumentet.
	\item[] \textbf{Afsnit 2 - SYSTEM OVERSIGT:} Dette afsnit giver en kort oversigt over systemet og dets omgivelser. Derudover vises der diagrammer over systemets akt�rer, samtidig med at systemet kort beskrives.
	\item[] \textbf{Afsnit 3 � SYSTEMETS GR�NSEFLADER:} Her beskrives gr�nsefladerne for de forskellige akt�rer til systemet.
	\item[] \textbf{Afsnit 4 � USE CASE VIEW:} Afsnittet beskriver alle Use Cases og Use Case scenarier fra Use Case modellen. De forskellige Use Cases beskrives i prosa form, samtidig med at et use case diagram pr�senteres. 
	\item[] \textbf{Afsnit 5 � LOGISK VIEW:} Beskriver systemets opdeling i delsystemer og pakker og deres organisering i en lagdelt struktur. Viser hvordan de forskellige Use Cases er realiseret, blandt andet ved brug af simple- og udvidede systemsekvensdiagrammer. 
	\item[] \textbf{Afsnit 6 � PROCES/TASK VIEW:} Dette afsnit beskriver systemets opdeling i processer og tr�de, og hvorledes disse processer kommunikerer og synkroniserer. Giver et overblik over de forskellige tr�de anvendt i systemet.
	\item[] \textbf{Afsnit 7 � DEPLOYMENT VIEW:} Viser den fysiske struktur af systemet, hvilket er computere og andre hardware enheder.
	\item[] \textbf{Afsnit 8 � IMPLEMENTERINGS VIEW:} Dette afsnit beskriver den endelige implementering. Her er de forskellige komponenter beskrevet i detaljer.
	\item[] \textbf{Afsnit 9 � DATA VIEW:} Er en beskrivelse af persistent data lagring i systemet. 
	\item[] \textbf{Afsnit 10 � GENERELLE DESIGNBESLUTNINGER:} Dette afsnit fastholder de generelle designbeslutninger, der tages under arktiekturdesignet. Her Beskrives systemets begr�nsninger, anvendte designm�nstre, samt anvendte v�rkt�jer og (kode)biblioteker.
	\item[] \textbf{Afsnit 11 � ST�RRELSE OG YDELSE:} I dette afsnit er angivet de kritiske st�rrelser og ydelsesparametre for systemet..
	\item[] \textbf{Afsnit 12 � KVALITET:} Her opremses de kvalitetskrav for systemet der er med til at udforme arkitekturen. Liges� beskrives hvorledes arkitekturen opfylder andre af de ikke-funktionelle krav vedr�rende for eksempel udvidbarhed.
	\item[] \textbf{afsnit 13 - OVERS�TTELSE:} Her beskrives der hvordan der kommes fra kildeteskt til objektkode, og hvordan systemet installeres
	\item[] \textbf{Afsnit 14 � K�RSEL:} Her beskrives hvordan de forskellige dele af systemet k�res, samt en forklaring p� de forskellige fejlbeskeder der kan forkomme.
	\item[] \textbf{Afsnit 15 � BILAG:} Liste over anvendte bilag.	 
	
	%\item[] \textbf{Afsnit 16 � Implementering og design:} Dette afsnit opdeler dele af systemet i komponenter og beskriver hvordan de er implementeret, og hvad der ligger til grund for implementationen.
	%\item[] \textbf{Afsnit 17 � Klassediagram og klassebeskrivelser:} Klassediagrammet vises og beskrives kort. Derefter pr�senteres klassebeskrivelseren
	%\item[] \textbf{Afsnit 18 � Brugermanual:} En manual til brugeren, der beskriver hvordan systemet skal benyttes.
\end{itemize}
\end{document}