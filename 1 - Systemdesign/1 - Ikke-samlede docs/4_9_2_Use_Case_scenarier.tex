\documentclass[main]{subfiles}
\begin{document}
\subsubsection{Use Case scenarier}
Denne Use Case har 3 normalforl�b, idet det b�de er muligt at tilf�je nyt stoppested til systemet, fjerne et stoppested fra systemet, samt �ndre i et stoppested der findes i systemet. Det er kun en administrator der kan initialisere denne Use Case.
Normalforl�b 1 beskriver hvorledes der kan tilf�jes et nt stopepsted til systemet. Dette forg�r ved at administratoren tilkendegiver overfor systemet at han �nsker at oprette et nyt stoppested. Systemet pr�sentere nu administratoren for et kort. Administratoren kan nu v�lge placering af stoppestedet p� kortet. N�r placering af stoppested er valgt, kan stoppestedet gemmes p� databasen, ved at brugeren tilkendegiver overfor systemet at stoppestedet �nskes gemt.

Normalforl�b 2 beskriver hvordan administratoren kan �ndre et allerede eksiterende stoppested.
Dette forg�r ved at administratoren v�lger et stoppested, fra listen over stoppesteder der findes i systemet. Administratoren kan nu �ndre placering samt navn for det valgte stoppested. �nskes �ndringerne at gemmes, kan administratoren tilkendegive overfor systemetat dette �nskes, hvorp� systemet vil gemme �ndringerne i databasen.

Normalforl�b 3 beskriver hvordan administratoren kan fjerne et allerede eksiterende stoppested fra systemet.
Dette forg�r ved at administratoren v�lger et stoppested, fra listen over stoppesteder der findes i systemet. Administratoren tilkendegiver nu overfor systemet at det valgte stoppested �nskes slettes fra systemet. Systemet sletter stoppestedet fra databasen. 

\subsubsection{Use Case undtagelser}
Da stoppestedet b�de skal gemmes p� en database, samt hentes fra en database, er der risiko for at forbindelsen til databasen mistes. Hvis dette sker, vil brugeren blive pr�senteres for en besked om at det ikke er muligt at etablere forbindelset til databasen.

\end{document}