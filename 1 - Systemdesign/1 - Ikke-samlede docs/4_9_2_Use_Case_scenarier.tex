\documentclass[main]{subfiles}
\begin{document}
\subsubsection{Use Case scenarier}
Denne Use Case har tre normalforl�b, idet det b�de er muligt at tilf�je, fjerne, samt �ndre et stoppested. Det er kun en administrator der kan initialisere denne Use Case.
Normalforl�b 1 beskriver hvordan der kan tilf�jes et nyt stoppested til systemet. Dette forg�r ved at administratoren tilkendegiver at et nyt stoppested �nskes oprettet. Systemet pr�senterer nu administratoren for et kort, hvorp� postionen for stoppested kan indtegnes. Herefter navngives det indtegnede stoppested, og det tilkendegives at dette �nskes gemt. Systemet gemmmer nu det oprettede stoppested p� den distribuerde database. \\

\noindent
Normalforl�b 2 beskriver hvordan administratoren kan �ndre et allerede eksiterende stoppested.
Dette forg�r ved at et allerede eksisterende stoppested v�lges fra listen af stoppesteder, hvorefter postionen og navn kan �ndres. Hvis �ndringerne �nskes gemt, kan administratoren tilkendegive dette, hvorp� systemet vil gemme �ndringerne i den distribuerede database.\\

\noindent
Normalforl�b 3 beskriver hvordan administratoren kan fjerne et allerede eksiterende stoppested.
Dette forg�r ved at et allerede eksisterende stoppested v�lges fra listen af stoppesteder, hvorefter der tilkendegives at dette �nskes slettet. Systemet sletter s� det valgte stoppested fra den distribuerde database.

\subsubsection{Use Case undtagelser}
Da informationen om stoppestederne b�de skal hentes fra og gemmes p� en den distribuerede database, er der risiko for, at forbindelsen til databasen mistes. Hvis dette sker, vil administratoren blive pr�senteret for en besked om, at det ikke er muligt at etablere forbindelse til databasen.

\end{document}