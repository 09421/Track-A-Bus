\documentclass[Main.tex]{subfiles} 
\begin{document}

\subsubsection{Use Case 8: rediger stoppested}
Use Case 8 best�r af 3 normalforl�b der omhandler at kunne tilf�je stoppesteder til database, slette stoppesteder fra databasen, og rediger position og navn for stoppesteder der allerede findes. alle 3 normalforl�b bliver startet fra \url{www.trackabus.dk/Stop}, Brugergr�nsenfladen kan ses p� figur\ref{fig:StopViewHjemmeside}. 

Normalforl�b 1 - Stoppested bliver tilf�jet til rute. Use Cases bliver initialiseret ved at administratoren trykker et sted p� det viste kort, dette vil trigger et klik event p� kortet er plasere en marker der hvor der blev trykket. Det er muligt at fjerne denne marker ved at trykker p� den, da det vil trigger et klik event p� markeren, der vil slette den, og derved g�re det muligt at plasere en ny marker. Efter at have plaseret markeren vil det v�re muligt at tr�kke den rundt p� kortet, for at plasere den pr�cis hvor den skal v�re. N�r den endelige �nskede plasering for markeren er fundet, indtastes navnet for stoppestedet i feltet 'Stop name'. Ved tryk p� knappen 'Save', vil gps-koordinaterne for markeren blive fundet, og blive sendt, sammen med stopnavnet, med et asynkron kald til StopController funktionen Save() der kalder ned i data tilgangs laget for at gemme det nye busstoppested i MySQL databasen.

Normalforl�b 2 bliver initialiseret ved at administratoren v�lger et stoppested fra listen over alle stoppesteder, og trykker p� knappen 'Delete'. Dette vil lave et asynkron kald til StopController funktionen Delete(), der kalder ned til data tilgangs laget, der sletter det valgte busstoppested fra MySQL databasen.

Normalforl�b 3 omhandler at kunne �ndre position og navn for et givet stoppested. Denne bliver initialiseret ved at administratoren v�lger et stoppested fra listen af stoppesteder, dette vil trigger et onchange event, der laver et asynkron kald til StopController funktionen GetPosistion() der kalder videre ned i data tilgangs laget, for at hente gps-koordinaterne for the valgte stoppested fra MySQL databasen. Koordinaterne vil blive sendt tilbage til viewet, der nu vil oprette en ny marker, med position p� kortet tilsvarende de modtaget koordinater. Administratoren har nu mulighed for at tr�kke i markeren for at �ndre dens position, og �ndre dens navn i feltet 'StopName'. N�r de �nskede �ndringer er fortaget, trykkes der p� knappen 'Save changes' der lavetet asynkron kald til StopController funktionen SaveChangeToStop(), der kalder videre ned i data tilgangs laget, der vil opdatere positionen og navnet for den valgte bus.

\end{document}