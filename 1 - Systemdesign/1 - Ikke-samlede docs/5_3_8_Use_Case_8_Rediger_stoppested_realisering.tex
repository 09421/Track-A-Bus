\documentclass[Main.tex]{subfiles} 
\begin{document}

\subsubsection{Use Case 8: Rediger stoppested}
Use Case 8 best�r af tre normalforl�b der omhandler at kunne tilf�je, fjerne eller �ndre stoppesteder i databasen. Alle tre normalforl�b bliver startet fra administrations hjemmesiden under stoppesteds redigering. \footnote{Administrations hjemmeside, stoppesteds redigering: \url{www.trackabus.dk/Stop}} 

\noindent
Normalforl�b 1 initialiseres ved, at administratoren trykker et sted p� det viste kort. Dette vil starte et click event p� kortet Der placerer en mark�r, hvor der blev trykket. Det er muligt at fjerne denne mark�r ved at trykker p� den igen, da dette vil trigger et click event p� mark�ren, som sletter den, og derved g�r det muligt at placere en ny mark�r. Efter at have placeret mark�ren, vil det v�re muligt at tr�kke den rundt p� kortet, s� dens position kan angives pr�cist. N�r den endelige �nskede plasering for mark�ren er fundet, indtastes navnet p� stoppestedet i feltet "Stop name". Ved tryk p� knappen "Save", vil gps-koordinaterne for mark�ren blive fundet, og sendt sammen med stoppestedsnavnet, med et asynkron kald til StopController funktionen "Save". Denne kalder til data tilgangs laget, hvor det nye busstoppested gemmes i databasen.\\

\noindent
Normalforl�b 2 bliver initialiseret ved at administratoren v�lger et stoppested fra listen af alle stoppesteder, og trykker p� knappen "Delete". Dette vil lave et asynkront kald til StopController funktionen "Delete", der kalder ned til data tilgangs laget, hvor s�rges for, at det valgte stoppested fjernes fra databasen.\\

Normalforl�b 3 omhandler at kunne �ndre position og navn for et givet stoppested. Denne bliver initialiseret ved, at administratoren v�lger et stoppested fra listen af alle stoppesteder. Dette vil starte et onchange event, der laver et asynkront kald til StopController funktionen "GetPosistion". Denne kalder videre ned i data tilgangs laget, for at hente gps-koordinaterne for det valgte stoppested fra databasen. Koordinaterne vil blive sendt tilbage til viewet, der nu vil oprette en ny marker p� kortet, med position tilsvarende det modtagede koordinat-s�t. Administratoren har nu mulighed for at tr�kke i markeren for at �ndre stoppestedets position, samt �ndre dennes navn i feltet "Stop Name". N�r de �nskede �ndringer er foretaget, trykkes der p� knappen "Save changes". Dette starter et asynkront kald til StopController funktionen "SaveChangeToStop, som kalder videre ned i data tilgangs laget. Her s�rges der for at stoppestedets navn og position bliver opdateret i databasen.

\end{document}