\documentclass[Main.tex]{subfiles} 
\begin{document}

\subsubsection{Use Case 6: Rediger bus p� rute}
Rediger bus p� rute Use Casen best�r af to normalforl�b, hvor det f�rste omhandler tilf�jelse af bus til en busrute, og normalforl�b 2 omhandler fjernelse af bus fra en busrute. Begge normalforl�b bliver startet fra administrations hjemmesiden under busredigering.\footnote{Administrations hjemmeside, bus redigering: \url{www.TrackABus.dk/bus}}.\\
Viewet vil, n�r siden er blevet indl�st, kalde funktionen "ExecuteOnLoad" der laver tre asynkrone kald til BusControlleren, der henter navne p� alle busruterne, alle busserne, og alle busser der ikke er knyttet til nogle rute. N�r disse er f�rdige vil viewet opdatere de relevante lister p� sk�rmen.

\noindent
Administratoren kan initialisere normalforl�b 1, ved at v�lge en busrute, fra listen af alle busruter. Ved tryk p� en af disse, vil et "onchange" event blive kaldt. Dette starter et asynkront kald til BusController funktionen "GetBussesOnRoute", som tilg�r data tilgangs laget for at hente alle de busser, der er knyttet til den valgte rute, i MySQL databasen. Busserne vil blive retuneret til BusControlleren, som igen returner tilbage til viewet, sim opdaterer listen "Busses on route" med de hentede busser.\\
Adminsitratoren kan nu tilf�je busser fra listen "Avaliable busses" til listen 'Busses on route', og trykke p� knappen 'Save'. Dette vil starte et asynkront kald til BusController funktionen "SaveChanges", der tilg�r funktionen i data tilgangs laget "SaveChangesToBus" bliver kaldt. Denne funktion knytter bussen til den valgte rute.\\

\noindent
Normalforl�b 2, forg�r p� samme m�de som i normalforl�b 1, med den undtagelse, at der bliver taget busser fra listen "Busses on route" og tilf�jet til listen "Avaliable busses". Efter der bliver trykket p� knappen "Save' vil der ske det samme som i normalforl�b 1, nu bare hvor busserne f�r fjernet deres knyttede ruter.
\end{document}