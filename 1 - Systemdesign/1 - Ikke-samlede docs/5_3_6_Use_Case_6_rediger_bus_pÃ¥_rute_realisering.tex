\documentclass[Main.tex]{subfiles} 
\begin{document}

\subsubsection{Use Case 6: Rediger bus p� rute}
Rediger bus p� rute Use Casen best�r af 2 normalforl�b, hvor det f�rste handler om at kunne tilf�je en bus til en busrute, og normalforl�b handler om at kunne fjerne en bus fra en busrute. Begge normalforl�b bliver startet fra \url{www.TrackABus.dk/bus}, Brugergr�nsenfladen kan ses p� figur \ref{fig:BusHjemmeside}. Viewet vil, n�r siden er blevet indl�st, kalde funktionen ExecuteOnLoad() der laver tre asynkrone kald til dens controller, der henter navne p� alle busruterne, alle busserne, og alle busser der ikke er knyttet til nogle rute. N�r disse er f�rdige vil viewet opdatere de relevante lister p� sk�rmen.

Administratoren kan initialisere normalforl�b 1, ved at v�lge en busrute, fra listen af alle busruter. Ved tryk p� en busrute, vil et onchange() event blive kaldt, der lavet et asynkron kald til BusController funktionen GetBussesOnRoute() der kalder ned i data tilgangs laget for at hente alle de busser fra MySQL databasen, der er knyttet til den valgte rute. Busserne vil blive retuneret tilbage til viewet, der vil opdatere listen 'Busses on route' med de hentede busser.\\
Adminsitratoren kan nu tilf�je busser fra listen 'Avaliable busses' til listen 'Busses on route', og trykke p� knappen 'Save'. Dette vil lave et asynkron kald til BusController funktionen SaveChanges(), der kalder videre til data tilgangs laget, hvor funktionen SaveChangesToBus() bliver kaldt, der knytter bussen til den valgte rute.

Normalforl�b 2, forg�r p� samme m�de som i normalforl�b 1, med den undtagelse at der bliver taget busser fra listen 'Busses on route' og tilf�jer dem til listen 'Avaliable busses'. Efter der bliver trykket p� knappen 'Save' vil der ske det samme som i normalforl�b 1, hvor den rute de er knyttet til vil blive sat til null.
\end{document}