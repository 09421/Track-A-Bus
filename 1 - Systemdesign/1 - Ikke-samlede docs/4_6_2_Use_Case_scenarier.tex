\documentclass[main]{subfiles}
\begin{document}
\subsubsection{Use Case scenarier}
Denne Use Case har tre normalforl�b, idet at man b�de kan tilf�je en bus, fjerne en bus eller �ndre i en eksistere bus.
Det er kun en administrator der kan initialisere denne Use Case.
Normalforl�b 1 beskriver, hvordan en bus tilf�jes til systemet. Dette forg�r ved at administratoren tilkendegiver overfor systemet at han vil tilf�je en bus til systemet. Herefter g�r systemet det muligt at indtaste information om bussen. N�r administratoren har indtastet det �nskede information, tilkendegiver administratoren at han �nsker at gemme informationen. Systemet gemmer nu informationen p� databasen.

Normalforl�b 2 beskriver hvordan administratoren �nder information om en bus der eksistere i systemet. Administratoren v�lger en bus fra listen over alle busser i systemet. Herefter tilkendegiver administratoren overfor systemet at han �nsker at �ndre information om den valgte bus. Systemet g�r det muligt for administratoren at �ndre information om bussen. N�r administratoren har indtastet det �nskede information, tilkendegiver administratoren at han �nsker at gemme informationen. Systemet gemmer nu informationen p� databasen

I normalforl�b 3 fjernes en bus. Denne initieres ved, at administratoren v�lger en bus fra en liste over alle busser i systemet. herefter tilkendegiver administratoren overfor systemet at han �nsker at fjerne den valgte bus fra systemet. Systemet fjerner nu den valgte bus fra databasen.

\subsubsection{Use Case Undtagelser}
Da informationen om busserne skal b�de hentes og gemmes p� en database, er der risiko for at forbindelsen til databasen mistes. Hvis dette sker, vil brugeren blive pr�senteres for en besked om at det ikke er muligt at etablere forbindelsetil databasen.


\end{document}