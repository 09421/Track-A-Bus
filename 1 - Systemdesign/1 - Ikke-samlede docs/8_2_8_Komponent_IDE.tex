\documentclass[Main.tex]{subfiles} 
\begin{document}

\subsubsection{Komponent 8: IDE}
Denne komponent g�r det muligt at skrive, gemme, hente og slette sekvenser til robotten ved hj�lp af en r�kke foruddefinerede funktioner, ydermere g�r den det muligt at tjekke den skrevet sekvens for syntaksfejl.
\\\\
\noindent
\textbf{Design}
\\
IDE'en er bygget op ved brug af IronPython og microsoft.scripting.
For at kunne k�re en skrevet sekvens, bruges der en IronPython script engine.
Denne er blevet implementeret via singleton pattern, da den samme engine skal bruges flere steder i systemet.
Denne engine benytter en string, fra en RichtextBox i ProgramsWindow klassen, og denne overs�ttes fra python kode til C\# kode der kan eksekveres p� robotten.
Til IDE'en er der implementeret en SyntaxIsOk() funktion som tjekker det f�rn�vnte script for python syntaks fejl.
Denne bliver brugt b�de n�r der trykkes p� Compile knappen, samt n�r programmet bliver eksekveret.
Der er blevet implementeret en r�kke funktioner som er mulig at kalde igennem IDE'en, disse funktioner er implementeret i InterfaceRobotClass klassen. funktionerne bliver brugt til styring af robotten, bla. OpenClaw(), GetWeight() og MoveByAxis() for mere information om disse funktioner, se \textit{SBS\_IDE-funktioner} documentet.
For at kunne kalde disse funktioner bliver der brugt en Dictionary\textless string, object\textgreater .
Dette giver mulighed for IronPython enginen at overs�tte en string, i dette tilf�lde "Robot"/"robot" til et givet objekt, igen i dette tif�lde et objekt af typen InterfaceRobotClass.
N�r IronPython enginen scripter den givne string, vil den overs�tte "Robot", fra en string til et objekt af typen InterfaceRobotClass.
Dette giver mulighed for, at i den overn�vnte string, at skrive f.eks Robot.OpenClaw, dette vil kalde OpenClaw funktionen implementeret i InterfaceRobotClass
    
\end{document}