\documentclass[Main.tex]{subfiles} 
\begin{document}

\subsubsection{Use Case 5: Rediger information om bus}
Denne Use Case best�r af 3 normalforl�b, der alle bliver initialiseret fra \url{www.TrackABus.dk/bus}, se figur \ref{fig:BusHjemmeside} for hvordan gr�nsefladen ser ud. Viewet vil, n�r siden er blevet indl�st, kalde funktionen ExecuteOnLoad() der laver tre asynkrone kald til dens controller, der henter navne p� alle busruterne, alle busserne, og alle busser der ikke er knyttet til nogle rute. N�r disse er f�rdige vil viewet opdatere de relevante lister p� sk�rmen.

Normalforl�b 1, hvor en ny bus skal tilf�jes til systemet bliver initialiseret, ved at administratoren indskriver ID'et for den nye bus i feltet: 'Bus ID', hvorefter der trykket p� knappen 'Add'. Dette vil tilf�je bussen til listen, men vil endnu ikke gemme det p� databasen, dette vil f�rst ske n�r administratoren trykker p� 'Save' knappen. Et tryk p� save knappen, vil kalde controller funktionen SaveBusChanges(), der vil finde de busser der er blevet tilf�jet til liste, samt de busser der er blevet f�rdig fra listen. SaveBusChanges() vil nu kalde funktionerne removeBusses() for at fjerne de busser fra databasen, der blev fjernet fra listen, og addBusses() for at tilf�je til databasen, de busser administratoren tilf�jede til listen. Disse to funktioner kalder ned i DBConnection klassen, der ligger i data tilgangs laget, hvilket er en static klasse, der bliver brugt til alt tilgang til MySQL databasen. Til sidste vil der blive retuneret til viewet, der nu vil informere administratoren om at der er blevet gemt.

Normalforl�b 2, hvor der skal slettes en bus fra MySQL databasen, bliver initialiseret ved at administratoren v�lger en bus, fra listen af alle busser gemt i databasen, hvorefter der bliver trykket p� knappen 'Remove'. Bussen vil nu blive fjernet fra listen. F�rst vil tryk op 'Save' knappen vil kalde funktionen SaveBusChanges() som i normal forl�b 1.

I normalforl�b 3, hvor det skal v�re muligt at �ndre information om en bus, bliver initialiseret ved at administratoren v�lger en af busserne fra listen, skriver et nyt ID i feltet 'Bus ID', hvorefter der bliver trykker p� knappen 'Rename'. Bussen vil nu skifte navn i listen, men igen f�rst ved tryk p� knappen 'Save' vil funktionen SaveBusChanges() blive kaldt, som i normalforl�b 1 og 2.
\end{document}