\documentclass[Main.tex]{subfiles} 
\begin{document}

\subsubsection{Use Case 5: Rediger information om bus}
Denne Use Case best�r af tre normalforl�b, der alle bliver initialiseret fra administrations hjemmesiden under busredigering.\footnote{Administrations hjemmeside, bus redigering: \url{www.TrackABus.dk/bus}}. Viewet vil, n�r siden er blevet indl�st, kalde funktionen "ExecuteOnLoad" der laver tre asynkrone kald til BusControlleren, der henter navne p� alle busruterne, alle busserne, og alle busser der ikke er knyttet til nogle rute. N�r disse er f�rdige vil viewet opdatere de relevante lister p� sk�rmen.

\noindent
Normalforl�b 1, hvor en ny bus skal tilf�jes til systemet bliver initialiseret, ved at administratoren indskriver ID'et for den nye bus i feltet: "Bus ID", hvorefter der trykkes p� knappen "Add". Dette vil tilf�je bussen til listen, men vil endnu ikke gemme det p� databasen. Dette vil f�rst ske n�r administratoren trykker p� "Save" knappen. Et tryk p� save knappen, vil kalde BusController funktionen "SaveBusChanges", der vil finde de busser, der er blevet tilf�jet eller fjernet i listen. "SaveBusChanges" vil nu kalde funktionerne "removeBusses" for at fjerne de busser fra databasen, der blev fjernet fra listen, og "addBusses" for at tilf�je til databasen, de busser administratoren tilf�jede til listen. Disse to funktioner kalder ned i DBConnection klassen, der ligger i data tilgangs laget, hvilket er en statisk klasse. Denne bliver udelukkende brugt som tilgang klasse til MySQL databasen. Til sidst vil der blive retuneret til viewet, der nu vil informere administratoren om at der er blevet gemt.

\noindent
Normalforl�b 2, hvor der skal slettes en bus fra MySQL databasen, bliver initialiseret ved at administratoren v�lger en bus, fra listen af alle busser gemt i databasen. Herefter bliver der trykket p� knappen "Remove" og bussen vil nu blive fjernet fra listen. F�rst ved tryk p� "Save" knappen vil funktionen "SaveBusChanges" blive kaldt som i normalforl�b 1.

\noindent
Normalforl�b 3, hvor det skal v�re muligt at �ndre informationen om en bus, bliver initialiseret ved, at administratoren v�lger en af busserne fra listen. Herefter indskrives nyt ID i feltet "Bus ID", hvorefter der bliver trykket p� knappen "Rename". Bussen vil nu skifte navn i listen, men igen f�rst ved tryk p� knappen "Save", vil funktionen "SaveBusChanges" blive kaldt, som i normalforl�b 1.
\end{document}