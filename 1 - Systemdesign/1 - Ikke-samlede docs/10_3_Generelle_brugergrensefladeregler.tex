\documentclass[Main.tex]{subfiles} 
\begin{document}

\subsection{Generelle brugergr�nsefladeregler}
Kravene til brugergr�nsefladen kan deles op i to kategorier:
\begin{itemize}
\item Arkitekturspecifikke
\item Udseendesspecifikke
\end{itemize}
Begge er beskrevet herunder
\subsubsection{Arkitekturspecifikke}
Brugergr�nsefalden er baseret p� MVVM designm�nsteret, som beskrevet i afsnit \textit{10.2 - Arkitektur m�nstre}. Dette er valgt af to grunde; Det betyder lavere kobling mellem model og view, og brugergr�nsefalden kan derfor udskriftes uden st�rre problemer. Desuden betyder det, at det er markant nemmere at teste kode der ligger t�t op af brugergr�nsefladen. 
Viewet er lavet udelukkende i XAML, og i viewmodellen ligger en r�kke properties, som omformer data fra modellen til viewet, samt en r�kke commands der kalder ned i modellens logik. Det var ideologen at der absolut ikke m�tte placeres kode i den s�kaldte "Code-behind" fil. Dette var dog ikke helt muligt n�r der f.eks. skulle �bnes nye vinduer. Derfor blev det besluttet, at s�danne operation skulle baseres p� Mediator Pattern (n�rmere beskrevet i afsnit \textit{10.2 - Arkitektur m�nstre}. Dette bet�d at der stadig blev holdt en forholdsvis lav kobling.

\subsubsection{Udseendesspecifikke}
For at holde et strengent ens udseende p� tv�rs af hele brugergr�nsefladen blev det besluttet at bruge WPF-themes fra codeplex. Dette p�virker at alle kontroller altid har samme udseende.

\end{document}