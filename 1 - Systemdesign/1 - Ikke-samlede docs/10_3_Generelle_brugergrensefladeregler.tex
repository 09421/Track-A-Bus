\documentclass[Main.tex]{subfiles} 
\begin{document}

\subsection{Generelle brugergr�nsefladeregler}
Kravene til brugergr�nsefladen kan deles op i to kategorier:
\begin{itemize}
\item Arkitekturspecifikke
\item Udseendesspecifikke
\end{itemize}
Begge er beskrevet herunder
\subsubsection{Arkitekturspecifikke}
Samtlige brugergr�nseflader er bygget op p� tre-lags modellen, som beskrevet i afsnit \textit{10.2: Arkitekturm�nstre}. Dette er gjort p� baggrund af, at brugergr�nsefladen nemt kan skiftes ud, hvis det skulle blive n�dvendigt, hvilket betyder en lavere kobling mellem data og pr�sentation. Hjemmesiden er bygget op p� baggrund af MVC, som er specifikt for ASP.NET. Dette g�r, ligesom tre-lags modellen, at hver komponent er udskiftelig, uden at der er behov for, at �ndringer foretages i viewet.

\subsubsection{Udseendesspecifikke}
Under applikations udvikling, blev det besluttet, at bruge to nuancer af r�d, til opbygning af gr�nsefladen. Disse to er baseret p� "midttrafik r�d", da det ville v�re dette firma, applikation skulle udvikles til, hvis et firma skulle kobles p� projektet.

\end{document}