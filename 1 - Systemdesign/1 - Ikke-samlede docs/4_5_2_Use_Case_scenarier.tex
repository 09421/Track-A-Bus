\documentclass[Main.tex]{subfiles} 
\begin{document}

\subsubsection{Use Case scenarier}
Denne Use Case viser et kort til burgeren, med indtegnet busrute, alle busser der k�re p� ruten, samt alle stoppesteder p� valgt rute. Det kr�ver at brugeren st�r ved listen over busruter, dern�st tilkendegiver brugeren overfor systemet hvilken busrute han �nsker vist. Derefter henter systemet busruten, samt stoppestederne p� ruten fra databasen.
Herefter bliver brugeren pr�senteret for et kort, med indteget busrute samt stoppesteder. Systemet henter nu gps-koordinaterne for busserne p� ruten samt indtegner dem p� kortet. Efter 2 sekunder vil systemet igen hente gps-koordinaterne, og opdatere bussernes position p� kortet. Systemet vil fors�tte med at opdatere bussernes position indtil brugeren tilkendegiver overfor systemet at dette ikke l�ngere �nskes. 


\subsubsection{Use Case Undtagelser}
Da busruten, stoppestederne samt bussernes gps-koordinater bliver hentet fra en database, er der risiko for, at forbindelsen til databasen mistes. Hvis dette sker, n�r systemet henter busruten og stoppestederne vil brugeren blive pr�senteres for en besked om at det ikke er muligt at etablere forbindelse til databasen, hvorp� han kan vende tilbage til startsk�rmen og pr�ve igen. Hvis det sker n�r systemet henter gps-koordinaterne vil brugeren blive pr�senteret for en besked om at det ikke er muligt at opdatere bussernes position. Der vil stadigv�k v�re muligt at se kortet, med indtegnet rute, samt stoppesteder, bussernes position vil blot ikke opdateres.
Det er mulighed for at systemet genetabler forbindelse til databasen, Hvis dette sker vil brugeren blive pr�senteret for en besked om at det igen er muligt at opdatere bussernes position. Systemet vil fors�tte med at opdatere bussernes position.


\end{document}