\documentclass[Main.tex]{subfiles} 
\begin{document}

\subsection{Exception og fejlh�ndtering}

\subsubsection{Fejlh�ndtering ved tab af internet forbindelse}
Da systemet er meget afh�ngigt af kommunikation med den distribuerede database, h�ndtere der i samtlige komponenter at forbindelsen til internet kan g� tabt.
\begin{itemize}
	\item Mobil applikationen s�rger for, at hvis internettet g�r tabt, bliver brugeren notificeret at der ikke er internet l�ngere, og derfor kan den givne funktion ikke l�ngere udf�res. Applikation s�rger for, at, hvis forbindelsen genetableres forts�ttes den givne funktion. Dette er mest relevant under opdatering af bussers position og/eller tid til valgt stop. En BroadcastReceiver bruges i sammeh�ng med at unders�ge forbindelsen til internettet. N�r der sker en �ndring i WIFI, eller i det mobile data netv�rk, vil denne unders�ge, om der efter den nye �ndring er tilgang til internettet. Et statisk flag s�ttes i klassen som symboliserer telefones forbindelse til netw�rk. Hvis denne er sat til false, vil data ikke hentes, og en fejlbesked vil vises.
	\item Simulatoren s�rger for, at brugeren bliver notificeret hvis der ikke er net, og pauser s�ledes systemet, indtil at nettet kan tilg�s igen.
	\item I tilf�lde af, at internettet forsvinder i mens en hjemmeside er indl�st, og en database-tilgangs funktion tilg�s, er der sat et timeout p� database p� to sekunder. Hvis dette timeout n�s, vil administratoren notificeres at internettet ikke kan tilg�s.
\end{itemize}


\end{document}
