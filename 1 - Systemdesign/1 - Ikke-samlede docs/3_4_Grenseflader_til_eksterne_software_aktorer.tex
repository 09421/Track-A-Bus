\documentclass[Main.tex]{subfiles} 
\begin{document}

\subsection{Gr�nseflader til eksternt software}
Programmet skal kommunikere med en dll-fil (Se \textit{Bilag: DLL filer/UDBC.dll}, for at se n�rmere p� denne), hvor forskellige funktioner til robotten findes. Denne dll-fil var udleveret sammen med robotten.

\subsubsection{USBC.dll}

I USBC.dll-filen findes alle funktioner der er n�dvendige, for at f� robotten til at bev�ge sig, som der �nskes. Funktionerne i biblioteket er dog skrevet i C++, og som det kan l�ses i \textit{SBS\_Kravspecifikation}, benyttes C\# som programmeringssprog. Dermed blev kommunikationen med denne en smule mere kompliceret, da C\# er managed code og C++ unmanaged code. C++ laver name mangling p� funktionerne, og derfor skulle funktionernes "specielle" navn findes. \\
Et andet problem var, at n�r en pointer (delegate) i C\# s�ttes til at pege p� noget data fra C++, kan garbagecollectoren ikke se, at der bliver peget p� noget. Dermed vil garbagecollectoren g� ind og deallokere pointeren. For at l�se dette problem er funktioner i marshall biblioteket blevet benyttet. Disse benyttes netop til at f� managed kode til at kunne kalde unmanaged kode fra en dll-fil. Dermed ignorerer garbagecollectoren pointerfunktionerne.
\\\\
De forskellige funktioner til robotten og deres tilh�rende beskrivelse er f�rst og fremmest fundet i dokumentet \textit{USBC-documentation}. Der g�res opm�rksom p�, at ikke alle funktioner er beskrevet fyldestg�rende i dette dokument.\\
N�r en funktion skal benyttes, findes den i dll-filen via visual studios kommandoprompt. Her benyttes kommandoens dumpbin p� dll-filen, og dermed listes samtlige funktioner med navn, samt hvordan navnet vil se ud i C\#. Disse er blevet gemt i et excel dokument (\textit{se bilag:  DLL filer/USBC excel}). Det underlige navn der ses, er hvordan funktionsnavnet ser ud i C\#.\\
Fra PInvoke benyttes 'DllImport'. Denne skal kende et EntryPoint, som er navnet p� funktionen i dll-filen (navnet som det vil se ud i C\# ). Dermed kan compileren kende navnet, og nu skal funktionen bare kaldes med en identisk prototype.\\
\begin{lstlisting}[caption=Funktionskald, label={lst:funkKald}]
[DllImport("USBC.dll", 
      EntryPoint = "?Initialization@@YAHFFP6AXPAX@Z1@Z",
        CallingConvention = CallingConvention.Cdecl)]
public static extern bool Initialization(
    short initMode,
    short systemType,
    CallBackFun.CallBackFunInitEnd fnInitEnd,
    CallBackFun.CallBackFunError fnErrMessage
    );
\end{lstlisting}
Kodeudsnit \ref{lst:funkKald} viser et eksempel p� et funktionskald fra dll'en. CallingConvention specificerer den m�de funktioner i unmanaged kode skal kaldes p�.\\
Alle funktionskald til dll-filen er lagt i en fil for sig (USBCDLL.cs). I parameterlisten er der to CallBack funktioner. Disse er pointere til andre funktioner. Alle disse CallBack funktioner, som er defineret som delegates, ligger ogs� i deres egen fil, s� der er en overskuelig opdeling (CallBackFun.cs).
\\\\
Udover de omtalte filer, skulle der ogs� laves to klasser, hvor der ligger noget information til de forskellige funktioner i dll'en. Den information funktionerne skulle bruge ligger i to headerfiler("USBCDEF", og "ERROR"), der ogs� er skrevet i C++ (se \textit{Bilag : DLL filer/UDBC.dll}). Disse skal naturligvis ogs� v�re identiske i vores projekt, og skal derfor konverteres til C\#.
\\
F�r klassen er der en attribut, som vist p� kodeudsnit \ref{lst:Attribut}. 
\begin{lstlisting}[caption=Titel der skal rettes, label={lst:Attribut}]
[StructLayout(LayoutKind.Sequential, Pack = 0, CharSet = CharSet.Ansi)]
\end{lstlisting}

Denne s�ger for at feltet i klassen bliver lagt i hukommelsen sekventielt, hvilket er vigtigt for at klassen kan bruges og forst�s. Derudover bruges [MarshalAs(..)] attributten, over array, string og andre typer der er forskel p� i C\# og C++. Klassen kan heller ikke bare initieres via new operatoren, da det vil se ud som om de ikke bruges, og dermed vil garbagecollectoren fjerne dem igen. Derfor skal de initieres p� en speciel m�de, som vist p� kodeudsnit \ref{lst:InitConfigData}
\begin{lstlisting}[caption=Titel der skal rettes, label={lst:InitConfigData}]
public void InitConfigData(ConfigData configData, IntPtr configDataPtr)
{
    configData = new ConfigData();

    configDataPtr = Marshal.AllocHGlobal(Marshal.SizeOf(typeof(ConfigData)));

    Marshal.StructureToPtr(configData, configDataPtr, false);

}
\end{lstlisting}
Via Marshal.AllocHGlobal() allokeres instansen, s� garbagecollectoren ikke fjerner den. Parameteren er st�rrelsen af det, der skal allokeres. Derefter S�ttes en pointer til at pege p� det.

\end{document}