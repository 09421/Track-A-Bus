\documentclass[Main.tex]{subfiles} 
\begin{document}

\subsubsection{Use Case scenarier}
Denne Use Case viser et kort til brugeren, med indtegnet busrute, stoppesteder samt busser der k�rer p� valgt rute. Det kr�ver at brugeren st�r ved listen over busruter, eller startsk�rmen, hvsi en rute er favoriseret. Dern�st tilkendegives der overfor systemet, hvilken busrute der �nskes vist. Systemet henter busruten, samt stoppestederne fra databasen.
Herefter bliver brugeren pr�senteret for et kort, med indteget busrute og stoppesteder. Systemet henter nu gps-koordinaterne for busserne p� ruten, samt indtegner dem p� kortet. Efter to sekunder vil systemet igen hente gps-koordinaterne, og opdatere bussernes position. Systemet vil fors�tte med at opdatere positionerne indtil brugeren tilkendegiver overfor systemet, at dette ikke l�ngere �nskes. 


\subsubsection{Use Case Undtagelser}
Da busruten, stoppestederne samt bussernes gps-koordinater bliver hentet fra en database, er der risiko for, at forbindelsen til databasen mistes. Hvis dette sker, n�r systemet henter busruten og stoppestederne, vil brugeren pr�senteres for en besked om, at det ikke er muligt at etablere forbindelse til databasen. Hvis dette sker n�r systemet henter gps-koordinaterne vil brugeren pr�senteres for en besked om, at det ikke er muligt at opdatere bussernes position. Det vil dog stadigv�k v�re muligt at se kortet, med indtegnet rute og stoppesteder, men bussernes position vil ikke l�ngere opdateres.
Der er mulighed for, at systemet genetabler forbindelse til databasen. Hvis dette sker, vil systemet igen opdatere bussernes position.


\end{document}