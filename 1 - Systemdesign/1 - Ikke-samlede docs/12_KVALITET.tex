\documentclass[Main.tex]{subfiles} 
\begin{document}

\section{KVALITET}

\subsection{Brugervenlighed}
Der er sat stort fokus p� brugervenlighed, da det er t�nkt som et distribueret system, der skal bruges af et mange forskellige personer. Dette er blevet opn�et ved, at der er tydelige knapper med sigende tekst og at brugeren skal g�re s� lidt som muligt for at opn� de resultat der �nskes.
\subsection{P�lidelighed}
Det er vigtigt at systemet er p�lideligt, da hovedfunktionaliteten i mobil applikationen er, at pr�cist kunne vise bussens position samt tiden til ankomst ved et valgt stoppested. Dette er blevet opn�et ved at bussens position bliver opdateret hvert sekund, og tiden bliver udregnet hver andet sekund. 
\subsection{Effektivitet}
Det er vigtigt at brugeren hurtigt kan f� vist en busrute samt tiden til ankomst for en ved et stoppested. Dette er blevet opn�et ved at g�re alle tunge udregninger og data processering p� en server, s� mobil telefonen ikke bliver belastet med det.
\subsection{Udvidbarhed}
Der er blevet sat et stort fokus p�, at hele systemet skal v�re nemt at videreudvikle p�, og vedligeholde. For at opn� dette der systemet bla. blevet opbygget p� tre-lags modellen, samt m�ngden af udregninger der sker p� serveren. Dette g�r det er nemt at udskifte de forskellige udregninger, og hvilken database der bliver brugt, uden at skulle opdatere selve mobil applikationen. Samtidig g�res det nemt at udvikle applikationen til en ny platform, da der ikke skal tages h�jde for database tilgang og udregninger.

\end{document}