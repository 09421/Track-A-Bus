\documentclass[Main.tex]{subfiles} 
\begin{document}

\section{KVALITET}

\subsection{Brugervenlighed}
Der er sat stort fokus p� brugervenlighed, da det er et distribuerede system der skal bruges af et utal af forskellige personer. Dette er blevet opn�et ved at der er tydelige knapper med sigende tekst og at brugeren skal g�re s� lidt som muligt for at f� vist en busrute og tiden til der ankommer en bus til et valgt stoppested.
\subsection{P�lidelighed}
Det er vigtigt at systemet er p�lideligt, da hovedfunktionaliteten i mobil applikationen er at pr�cist kunne vise bussens position samt tiden til ankomst ved et valgt stoppested. Dette er blevet opn�et ved at bussens position bliver opdateret hvert sekund, og tiden bliver udregnet hver andet sekund. 
\subsection{Effektivitet}
Det er vigtigt at brugeren hurtigt kan f� vist en busrute samt tiden til bus ankommer til stoppested. Dette er blevet opn�et ved at g�re alle tunge udregninger og data processering p� en server, s� mobil telefonen ikke bliver belastet med det.
\subsection{udvidbarhed}
Der er blevet sat et stort fokus p� at hele systemet skal v�re nemt at videreudvikle p�, og vedligeholde. For at opn� dette der systemet bla. blevet opbygget p� ved brug af 3-lags-modellen, og ved at meget af arbejdet sker p� en service. dette g�r det er nemt at udskifte de forskellige udregninger, og hvilken database der bliver brugt, uden at skulle opdatere selve mobil applikationen.

\end{document}