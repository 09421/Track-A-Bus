\documentclass[Main.tex]{subfiles} 
\begin{document}

\section{KVALITET}

\subsection{Brugervenlighed}
Det er tilstr�bt at holde systemet simpelt og brugervenligt som muligt. Dette er gjort ved at lave forholdsvis store knapper med sigende tekst. 

\subsection{P�lidelighed}
Der er lidt problemer mht. systemets p�lidelighed, da der ofte opst�r problemer mht. at home robotten. Men da dette enten skyldes funktioner i .dll-filen eller selve hardwaren, har det ikke v�ret muligt fuldst�ndig at udrette dette problem. Til geng�ld er systemet lavet s�ledes, at brugeren l�bende har mulighed for at home robotten, hvis der skulle opst� fejl. S�ledes er det ikke n�dvendigt at genstarte hele programmet, hvis der skulle opst� fejl p� denne front. Desuden at der lavet en del fejlh�ndtering, beskrevet n�rmere under afsnit \textit{10.4 Exceptions og fejlh�ndtering}.

\subsection{Integritet}
Den fysiske n�dstopknap afbryder str�mme til systemet, og dette resultere naturligvis i at robotten stopper i sin respektive position. Desuden er det ogs� muligt at stoppe en igangv�rende sortering gennem brugergr�nsefladen. Ligeledes at det muligt at s�tte en sorteringsmekaniske p� pause, og starte denne igen.

\end{document}