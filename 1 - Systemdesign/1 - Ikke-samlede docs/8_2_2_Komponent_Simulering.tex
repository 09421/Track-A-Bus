\documentclass[Main.tex]{subfiles} 
\begin{document}

\subsubsection{Komponent 2: Simulering}
Denne komponent har til form�l at simulere robotten.
Der er nogle gode grunde til at simulere robotten udover den, at selve produktopl�gget kr�ver en simulering:
\begin{itemize}
\item F�rst og fremmest skal man v�re opm�rksom p�, at der kun er en robot til r�dighed. Dermed er det kun en begr�nset tid man har adgang til robotten, hvormed det kan blive sv�rt at teste ens programmer.
\item Derudover tager det minimum to minutter at home robotten, hvilket ofte vil v�re en n�dvendighed for at teste sit program. Dermed vil den kostbare tid man har ved robotten, blive formindsket endnu mere.
\end{itemize}
Med disse ulemper taget i betragtning, er det klart, at en simulering vil v�re en stor fordel.\\\\
\textbf{Design:}\\
Simulatoren best�r af forskellige simulerings-klasser. Disse oprettes og bruges via constructor-injection p� programmet. Der er blevet taget udgangspunkt i, at man skal kunne simulere ens program uden nogen kontakt til robot eller database. Dermed har det v�re n�dvendigt at lave f�lgende klasser:
\begin{itemize}
\item RoboSimulationCalls (implementerer IRoboCalls. Heri ligger samtlige funktioner der bruges til robotten)
\item DataBaseSimulation (implementerer IDatabaseFacade)
\item DatabaseObserverSimulering (implementerer IObserver)
\item FakeWeight (implementerer IWeight)
\item FakeRobotEvent (implementerer IRobotEvent)
\end{itemize}
Forskellen p� simulatorerne og fake klasserne, er at simulatorerne logger nogle strenge med information over hvad der er sket, mens fake klasserne s�tter nogle variable, der s�ger for at programmet kan k�re uden robotten. Dermed kan man k�re det rigtige program, i samspil med robotten, eller n�jagtig det samme program som en simulering. Dette sikres som n�vnt via constructor-injection.\\
En anden vigtig del af simulatoren er loggen. Ud fra log interfacet er der implementeret tre forskellige logningsmetoder, hvor simuleringen bruger to af dem. Simuleringen logger til et vindue, samt til en fil der ligger i en mappen Simuleringslogs i rodmappen (RobotProjekt mappen)\\
Diagrammerne i afsnit \textit{5.3.7 Use Case 5} kan give et overblik over simulatoren.\\
\textbf{RoboSimulationCalls klassen:}\\
Herunder ses et kodeudsnit fra en af funktionerne i RoboSimulationCalls:
\begin{lstlisting}[caption=Funktionen Initialization(...)]
public bool Initialization(short sMode, short sSystemType)
{
    string sModeS;
    string sSystemTypeS;
    switch (sMode)
    {
        case 0:
            sModeS = "INIT_MODE_DEFAULT selects last used mode (from ini file).";
            break;
        case 1:
            sModeS = "INIT_MODE_ONLINE force online mode.";
            break;
        case 2:
            sModeS = "INIT_MODE_SIMULAT selects simulation mode.";
            break;
        default:
            sModeS = "Unknown INIT_MODE.";
            break;
    }
    switch (sSystemType)
    {
        case 0:
            sSystemTypeS = "DEFAULT_SYSTEM_TYPE let the libary detect the robot type.";
            break;
        case 41:
            sSystemTypeS = "ER4USB_SYSTEM_TYPE define robot type as ER-4 Scorbot with USB connection.";
            break;
        default:
            sSystemTypeS = "Unknown SYSTEM_TYPE.";
            break;
    }

    logWindow.LogThis(loggingLevels.INFO, "Initialization: " + retVal, false);
    logWindow.LogThis(loggingLevels.INFO, "Initialization MODE: " + sModeS + " ## " + sSystemTypeS, true);

    logFile.LogThis(loggingLevels.INFO, "Initialization: " + retVal, false);
    logFile.LogThis(loggingLevels.INFO, "Initialization MODE: " + sModeS + " ## " + sSystemTypeS, true);
    logFile.LogThis(loggingLevels.INFO, "newline", false);

    initRuned = true;
    return retVal;
}
\end{lstlisting}
Ovenst�ende eksempel f�lger den m�de, hvorp� simulatoren er opbygget. Den tager parametre, som svarer til et mode eller lignende, og genererer et tekststreng ud fra det. Hvilket mode det svarer til er fundet i USBC-documentationen. Tekststrengen bliver derefter logget. Derudover er funktionerne opbygget s�ledes at de alle returnerer en bool. Denne returv�rdi er i simuleringssammenh�ng en bool, som man selv kan s�tte v�rdien p�.\\
Som det ses i eksemplet s�ttes et private variabel 'initRuned'. Denne fungerer som validering. Hvis andre funktioner kaldes f�r denne er sat, f�r brugeren af vide, at robotten ikke er initialiseret. Udover denne s�ttes en variabel i Control og Home funktionen ogs�, da disse er essentielle for at mange af robotfunktionerne kan kaldes.\\\\
\textbf{DatabaseObserverSimulering klassen:}\\
Denne klasse er som s�dan ikke en simulering. Den har samme funktionalitet som den rigtige version, men undlader funktionalitet til at h�ndtere en tabt forbindelse til databasen.\\\\
\textbf{DataBaseSimulation klassen:}\\
Denne h�ndterer kaldene fra DatabaseObserverSimulering. Den er opbygget efter samme princip som hovedsimulatoren "RoboSimulationCalls", hvor en streng, opbygget ved hj�lp af input, logger til konsol og fil.

\end{document}