\documentclass[Main.tex]{subfiles} 
\begin{document}

\subsubsection{Use Case 3: Vis tid for n�rmeste bus for valgt stoppested}

Use casen kan initieres efter Use Case 2 er fuldf�rt, og et kort er vist med ruter og stoppesteder, men opdaterende busser. N�r et stoppested trykkes, s�ttes stoppestedsnavnet i den �verste informations bar p� viewet. Desuden s�ttes teksten i begge endestationsbokse til "Loading", og tiden s�ttes til "nn:nn:nn". Herefter unders�ges der f�rst om der er net. Hvis der er, unders�ges der om et stoppested allerede er valgt, og dermed om Use Case 3 allerede er igang, og hvis der er, interruptes tr�den. N�r tr�den er helt lukket, kaldes GetBusTime i TrackABusProvideren med ruten, det valgte stoppested samt en ny MessageHandler.\\
GetBusTime st�r for at hente tiden og endestationen for den n�rmeste bus, i begge retninger p� ruten. Udregningen af dette ligger som en stored procedure p� MySQL databasen. Dette kan l�ses om i afsnit \textit{9.1.2: Stored Procedures}. N�r udregningerne er f�rdige, returneres resultaterne til TrackABusProvideren, hvor de bliver pakket ned i en Message, og sendt til MessageHandleren. MessageHandleren ligger i BusMapActivity, og s�rger for at s�tte resultaterne, i de rigtige views. Hvis der ikke er nogen bus, som k�rer forbi stoppestedet i en eller begge retninger, retuneres der "anyType{}" i den relevante retnings endesstations v�rdi. Hvis en eller begge endestations resultater er denne v�rdi, opdateres endestations teksten til  "No bus going in this direction" og tiden til "nn:nn:nn", for den relevante retning. Hvis endestations teksten i forvejen er "No bus going in this direction", opdateres der der intet for den givne retning. Grunden til endestations v�rdien opdateres er, at en rute kan v�re kompleks med mere end to stoppesteder, og en given bus, kan have k�re p� samme ruten som en anden bus, men ikke have samme endestation.\\
\\Hvis internettet ikke kan tilg�s under protokollen, vil tr�den stadig k�re, men der vil blive vist en besked om, at tiden ikke l�ngere holdes opdateret. Herved, n�r forbindelsen genskabes, forts�ttes tidsopdateringen p� ny. N�r BusMapActivity g�r i dvale s�ttes tidsopdaterings tr�den p� pause, og n�r servicen bindes igen, ved gen�bning, startes processen igen.
P� \ref{fig:UC3SSD} kan et simpelt sekvensdiagram, for forl�bet ses. Den viser kun en hentning, som antages at v�re succesfuld.

\begin{figure}[H]
	\centering
	\includegraphics[scale=0.65]{Diagrammer/Sekvensdiagrammer/UC3SSD}
	\caption{Sekvensdiagram for hentning af endestationer og tid for n�rmeste bus.}
	\label{fig:UC3SSD}
\end{figure}




\end{document}