\documentclass[Main.tex]{subfiles} 
\begin{document}

\subsection{System introduktion}
Systemet best�r af 2 hoved dele, en mobil applikation og en administrations hjemmeside.\\ Mobil applikationens hovedfunktion er at g�re det muligt for brugeren at se en valgt busrute, dens busstoppesteder og alle busser der k�re p� ruten, indtegnet p� et kort, og f� vist hvor lang tid der er til den n�ste bus ankommer til et valgt busstoppested. Det er ogs� mulighed for at favorisere de forskellig busruter, dette vil gemme busruten lokalt p� mobil telefonen, og derfor skal der ikke bruges mobil data til at downloade ruten hver gang den skal bruges.\\ Hoved funktionaliteten for administrations hjemmesiden er at g�re det nemt at tilf�je en ny busruter med stoppesteder til systemet, samt g�re det muligt at �ndre allerede eksisterende busruter. Alt dette sker igennem en GUI best�endene af hjemmesiden \url{www.TrackABus.dk}.
\\
\\
Desuden er der ogs� blevet udviklet en simulator der har til form�l at simuler en eller flere busser der k�re p� de forskellig busruter. Simulatoren er ikke en del af hovedsystemet, og derfor ikke beskrevet i Use Cases. Det er muligt at tilg� simulatoren igennem en GUI, hvorfor en udvikler kan starter forskellige simulationer, s� som en enkel bus der k�re en bestemt vej p� ruten, alle busser p� en rute, eller alle busser i systemet p� deres rute. Det er n�dvendig med en simulator da der skal v�re mulighed for at teste forskellige situationer og algoritmer, hvor man ikke kan v�re afh�ngig af en rigtig bus. \textit{Se afsnit 6.2.4 Komponent 4: Simulator} for dybere beskrivelse af simulatoren.
\end{document}