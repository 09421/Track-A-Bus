\documentclass[Main.tex]{subfiles} 
\begin{document}

\subsection{System introduktion}

Systemet hovedfunktion er, at sorterer klodser efter materialetype, samt lade en programm�r programmere egne programmer til systemet.
\\
Programm�ren tilg�r systemet vha. en GUI og har herefter mulighed for at starte systemets forskellige programmer. Derudover har han mulighed for at skrive egne programmer, for at tilg� databasen og redigere i materialetyper og klodser samt mulighed for at indstille placeringen af transportb�ndet klodserne kommer k�rende p�.\\
Hvis den rigtige robot ikke skulle v�re tilg�ngelig, er der mulighed for at simulere programmerne, hvorved alle informationer om hvad robotten ville g�re, bliver skrevet ud p� sk�rmen samt lagt i en logfil. Disse logfiler kan naturligvis tilg�s. Der bliver ogs� oprettet logfiler, n�r ens program k�res med den rigtige robot. Disse kan naturligvis ogs� tilg�s.
\\
\\
Ligeledes skal systemet kunne betjenes af en operat�r, som dog ikke har mulighed for at omprogrammere robotten. Han har kun mulighed for at k�rer standardprogrammet (der sorterer klodserne efter materialetype), eller k�rer et af de brugerdefinerede programmer programm�ren har lavet. Derudover har operat�ren ogs� mulighed for at til logfiler.\\
Sidst skal det n�vnes at b�de operat�r og programm�r har mulighed for at printe lister over materialer og klodser.
\end{document}