\documentclass[Main.tex]{subfiles} 
\begin{document}

\subsection{System introduktion}
Systemet best�r af to hoveddele, en mobil applikation og en administrations hjemmeside.\\ Mobil applikationens hovedfunktion er at g�re det muligt for brugeren at se en valgt busrute, rutens stoppesteder og alle busser der k�rer p� ruten, indtegnet p� et kort. Desuden er det muligt at f� vist, hvor lang tid der er, til den n�ste bus ankommer til et valgt stoppested. Det er ogs� muligt at favorisere de forskellig busruter. Dette vil gemme busruten lokalt p� mobil telefonen, hvorefter der ikke l�ngere bruges mobil data til at downloade ruten, hver gang den skal bruges.\\ Hoved funktionaliteten for administrations hjemmesiden er at g�re det nemt at tilf�je en ny busruter med stoppesteder til systemet, samt g�re det muligt at �ndre allerede eksisterende busruter. Alt dette sker igennem en GUI p� TrackABus hjemmsiden.\footnote{Administrationsside: \url{www.TrackABus.dk}}
\\
\\
Der er ogs� blevet udviklet en simulator der har til form�l at simulere en til flere busser, der k�rer p� de forskellig busruter. Simulatoren er ikke en del af hovedsystemet da den kun er lavet til udviklingsform�l, og derfor ikke beskrevet i Use Cases. Det er muligt at tilg� simulatoren igennem en GUI, hvorfra en udvikler kan starte forskellige simulationer. Det er n�dvendigt med en simulator, da der skal v�re mulighed for at teste forskellige situationer og algoritmer, hvor man ikke kan v�re afh�ngig af en rigtig bus. \textit{Se afsnit 8.2.4 Komponent 4: Simulator} for en dybere beskrivelse af simulatoren.
\end{document}