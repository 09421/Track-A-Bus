\documentclass[Main.tex]{subfiles} 
\begin{document}

\subsection{Arkitektur m�nstre}

Nedenst�ende liste er en opremsning af de standard arkitekturm�nstre, der er anvendt i systemet. 
Under listen er de forskellige m�nstre dokumenteret.

\begin{itemize}
  \item Model View ViewModel (MVVM)
  \item Singleton
  \item Adapter
  \item Strategy
  \item Facade
  \item Publish-Subscribe
  \item Mediator
\end{itemize}

\subsubsection*{Model View ViewModel (MVVM)}

MVVM-m�nstret bruges til at opretholde en overskuelig arkitektur i systemets implementering. 
Det s�rger bl.a. for, at det er muligt at udf�re softwaretest p� alt kode.
Koden opdeles i fire lag; \textit{View}, \textit{ViewModel}, \textit{Model} og \textit{DAL}\footnote{Data acces Layer}. 
\\
Se \url{http://en.wikipedia.org/wiki/Model_View_ViewModel} for mere dokumentation.



\subsubsection*{Singleton}
Singleton bruges til at oprette �t objekt af en instance. 
N�r f�rst objektet er oprettet vil andre, der fors�ger at oprette det, f� returneret det samme objekt, som der blev oprettet f�rst.
\\
Se afsnit 26.5 i \textit{Applying UML and Patterns} for yderlige information.



\subsubsection*{Adapter}
Adapter-m�nstret bruges til at oprette et interface og en klasse til en anden klasse, s�ledes den anden klasses funktioner kan tilg�s fra interfacet.
\\
Se afsnit 26.1 i \textit{Applying UML and Patterns} for yderlig information.



\subsubsection*{Strategy}
Strategy-m�nstret bruges til at udskifte funktionalitet p� run-time, s�ledes den samme funktion kan kaldes, men hentes fra forskellige klasser.
\\
Se afsnit 26.7 i \textit{Applying UML and Patterns} for yderlig information.



\subsubsection*{Facade}
Facade-m�nstret bruges til at pakke flere sammenh�ngende klasser ind i en .dll-fil, s�ledes det kan tilg�s af vilk�rligt andre klasser med samme funktionalitet.
\\
Se afsnit 26.9 i \textit{Applying UML and Patterns} for yderlig information.



\subsubsection*{Publish-Subscribe}
Publish-Subscribe-m�nstret bruges til at f� flere klasser til at videregive information, uden de skal vente p� en returkald. 
P� denne m�de kan andre klasser tilg� informationen, uden klassen der afgav den informationen kender de(n) der modtager.
\\
Se afsnit 26.9 i \textit{Applying UML and Patterns} for yderlig information.



\subsubsection*{Mediator}
Mediator-m�nstret opretter en klasse til at sende information imellem to andre klasser, s�ledes kaldet kan testes.
\\
Se \url{http://en.wikipedia.org/wiki/Mediator_pattern}


\end{document}