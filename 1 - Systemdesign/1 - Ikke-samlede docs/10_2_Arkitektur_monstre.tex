\documentclass[Main.tex]{subfiles} 
\begin{document}

\subsection{Arkitektur m�nstre}

Nedenst�ende liste er en opremsning af de arkitekturm�nstre, der er anvendt i systemet. 
Under listen er de forskellige m�nstre dokumenteret.

\begin{itemize}
  \item Tre-lags model
  \item MVC (Model-View-Controller)
  \item ContentProvider
  \item Bound Service
\end{itemize}

\subsubsection*{Tre-Lags model}

Tre-lags modellen bruges i sammenh�ng med seperation-of-concerns, til at splitte et projekt op i pr�sentation, logik og data tilgang. Pr�sentations laget s�rger for at h�ndtere alt visuelt, logik laget laver manipulationer p� data, og data tilgangen s�rger for at tilg� de eksterne data kilder og hente relevant data. 
\\
Se \url{http://en.wikipedia.org/wiki/Multitier_architecture} for mere dokumentation.

\subsubsection*{MVC}
En afart af tre-lags modellen og bruges i sammenh�ng med hjemmeside design i ASP.NET. Viewet pr�senterer data for brugeren, hentet fra modellen. Brugeren har ogs� mulighed for at tilg� controlleren igennem viewet og �ndre data sat i model-laget. Efter kaldet, vil viewet blive opdateret med de nye �ndringer.
\\
Se \url{http://en.wikipedia.org/wiki/Model-view-controller} for mere dokumentation.

\subsubsection*{ContentProvider}
En Android specifik metode til at tilg� data. Bruges i sammenh�ng med data-deling, mellem applikationer. Ved at bruge forskellige URI'er, kan et kald til samme funktion, tilg� forskellige funktionaliteter.
\\
Se \url{http://developer.android.com/reference/android/content/ContentProvider.html} for mere dokumentation.

\subsubsection*{BoundService}
Et Android specifikt interface til en client/server implementering. N�r en service er bound, kan den kun tilg�s i den givne applikation, fra de klasser der binder til den. Efter service kaldet er f�rdiggjort, returneres der til den kaldende klasse, over en message handler.
\\
Se \url{http://developer.android.com/guide/components/bound-services.html} for mere information.




\end{document}