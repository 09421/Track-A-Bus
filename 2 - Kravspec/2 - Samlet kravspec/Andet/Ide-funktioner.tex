\documentclass[12pt,a4paper]{article}
\usepackage[Danish]{babel}
\usepackage{amsmath}
\usepackage{amsfonts}
\usepackage{amssymb}
\usepackage{graphicx}
\usepackage[utf8]{inputenc}

\begin{document}
\section*{IDE Funktioner}
For at bruge de givet robot funktioner skal der skrives Robot. hvorefter en liste af de mulige funktioner vil blive vist 
\\

\subsection*{Init Functions}

\textbf{Navn:}
\\bool InitRobot()
\\
\textbf{Beskrivelse:}
\\Denne function initializer robotten, samt g�r det muligt at bruge de resterende Robot funktioner
\\Denne function skal altid kaldes som
det f�rste i et nyt program
\\
\\
\textbf{Navn:}
\\bool ResetRobot()
\\
\textbf{Beskrivelse:}
\\Denne function vil flytte robotten tilbage til start position

\subsection*{Move Functions}
\textbf{Navn:}
\\bool MoveByAxis(Int32 x, Int32 y, Int32 z, Int32 p, Int32 r)
\\Bev�ger robotten linear, relativ til den nuv�rende position
\textbf{Beskrivelse:}
\\
\\
\textbf{Navn:}
\\bool MoveByCoordinates(Int32 x, Int32 y, Int32 z, Int32 p, Int32 r)
\\
\textbf{Beskrivelse:}
\\Bev�ger robotten linear ud fra de givet koordinater
\subsection*{Functions}
\textbf{Navn:}
\\void OpenClaw()
\\Bev�ger robotten
\textbf{Beskrivelse:}
\\�bner Robot kloen helt
\\
\\
\textbf{Navn:}
\\void CloseClaw()
\\
\textbf{Beskrivelse:}
\\Lukker Robot kloen helt
\\
\\
\textbf{Navn:}
\\bool SetClawValue(short millimeter)
\\
\textbf{Beskrivelse:}
\\S�tter kloen til den givet v�rdi i millimeter
\\
\\
\textbf{Navn:}
\\int GetClawValue()
\\
\textbf{Beskrivelse:}
\\Returnerer v�rdien af kloen i millimeter   
\\
\\
\textbf{Navn:}
\\double GetWeigth()
\\
\textbf{Beskrivelse:}
\\Returnere v�rdien fra v�gten, i gram

\subsection*{Other functions}
Disse funktioner bliver kaldt p� anden m� de end de �vrige
\\Se relevante beskriverlser for mere info
\\
\\
\textbf{navn:}
\\MessageBox.Show(string)
\\
\textbf{Beskrivelse:}
\\Viser en MessageBox med en givet string
\textit{\\eksempel: msg.MessageBox("Test")}
\\
\end{document}