\subsection{Use Case 1: Vis busruter}
\textbf{M�l:}
\\
M�let med denne Use Case er at f� vist alle distribuerede persisterede busruter. 
\\
\\
\textbf{Initiering:}
\\
Brugeren tilkendegiver over for systemet, at han �nsker at f� vist alle distribuerede persisterede busruter. 
\\
\\
\textbf{Akt�rer og interessenter:}
\\
Prim�re akt�rer:
\begin{itemize}
\item Bruger.
\end{itemize}
\textbf{Antal samtidige forekomster:}
\\
En samtidig forekomst.
\\
\\
\textbf{Ikke funktionelle krav:} \\
Ingen\\

\noindent
\textbf{Startbetingelser:}
\\
Programmet er startet op, og brugeren st�r ved startsk�rmen.(Se figur \ref{fig:MainScreen})
\\
\\
\textbf{Slutresultat ved succes:}
\\
Brugeren f�r vist en liste over alle distribuerede persisterede busruter.(Se figur \ref{fig:RouteListScreen})
\\
\\
\textbf{Slutresultat ved undtagelser:}
\\
Brugeren f�r vist startsk�rmen.
\\
\\
\textbf{Normalforl�b A:}

\begin{enumerate}
\item Systemet tilg�r distribueret persisteret data, og henter busruterne.
\item Systemet pr�senterer brugeren overfor listen af busruter.
\end{enumerate}
\textbf{Undtagelser:}
\begin{enumerate}[\text{Undtagelse} A:] 
\item Persisteret data kan ikke tilg�s.

	\begin{enumerate}[1.]
	\item Systemet viser en fejlmeddelelse til brugeren, der beskriver, at det ikke er muligt at indl�se busruterne.
	\item Systemet returnerer til startsk�rmen.
	\end{enumerate}
\end{enumerate}

\begin{enumerate}[\text{Undtagelse} B:] 
\item Brugeren annullerer indl�sningen. 
	\begin{enumerate}[1.]
	\item Systemet annullerer indl�sningen af data.
	\item Systemet returnerer til startsk�rmen.
	\end{enumerate}
\end{enumerate}

\begin{enumerate}[\text{Undtagelse} C:] 
\item Systemet g�r i dvale, under indl�sningen. 
	\begin{enumerate}[1.]
	\item Systemet indl�ser busruterne f�rdig i baggrunden.
	\end{enumerate}
\end{enumerate}