\subsection{Use Case 2: Vis placering af alle busser og stoppesteder p� valgt rute}
\textbf{M�l:}
\\
M�let med denne Use Case er at f� vist alle busser samt busstoppesteder, for en valgt rute, p� kortet. Busstoppesteder og busser vil v�re tydeligt markeret.
\\
\\
\textbf{Initiering:}
\\
Brugeren tilkendegiver over for systemet, hvilken busrute han �nsker f� vist p� kortet.
\\
\\
\textbf{Akt�rer og interessenter:}
\\
Prim�re akt�rer:
\begin{itemize}
\item Brugeren
\end{itemize}
\textbf{Antal samtidige forekomster:}
\\
En samtidig forekomst.
\\
\\
\textbf{Ikke funktionelle krav:}
\begin{enumerate}[A.]
	\item Opdatering af busposition forekommer hvert sekund, med en max. afvigelse p� +0.5 sekunder
\end{enumerate}
\textbf{Startbetingelser:}
\\
Initialisering kr�ver, at en af normalforl�bene for f�lgende Use Cases er fuldendt:
\begin{itemize}
\item Use Case 4 - Tilf�j/Fjern busnummer til listen af favoritter.
\begin{itemize}
\item Hvis en favoritrute i forvejen er tilf�jet, kan Use Case 2 startes fra hovedsk�rmen.
\end{itemize}
\item Use Case 1 - Vis liste af busruter
\begin{itemize}
\item Alle busruter pr�senteres, efter fuldendt normalforl�b for Use Case 1.
\end{itemize}
\end{itemize}
\textbf{Slutresultat ved succes:}
\\
Brugeren vil blive pr�senteret for et kort, hvorp� busruten er tegnet ind, med tydeligt markerede busstoppesteder. P� ruten vil alle k�rende busser, deres retning samt busstoppestederne blive tydeligt vist.
\\
\\
\textbf{Slutresultat ved undtagelser:}
\\
Brugeren f�r vist kortet med indteget busrute.
\\
\\
\textbf{Normalforl�b A:}

\begin{enumerate}[1.]
\item Systemet �bner og vister Kortet.
\item Systemet henter busruten og stoppestederne fra det distribuerede persisteret data.
\item Systemet indtegner busruten og stoppestederne p� kortet.
\item Systemet henter positionen for busserne p� den valgte rute.
\item Systemet indtegner busserne p� deres position.
\item Systemet vil efter et tidsinterval, hente bussernes position igen.
\item Systemet opdatere bussernes position p� kortet.
\item Systemet g�r til punkt 5.
\\
\end{enumerate}
\textbf{Undtagelser:}
\begin{enumerate}[\text{Undtagelse} A:] 
\item Hentningen af data annulleres. 
	\begin{enumerate}[1.]
	\item Systemet stopper med at hente bussernes positions data.
	\item Sysemet returner til listen over busruter.
	\end{enumerate}
\end{enumerate}
\begin{enumerate}[\text{Undtagelse} B:] 
\item Systemet g�r i dvale, under hentning af data. 
	\begin{enumerate}[1.]
	\item Hentningen g�res f�rdige, og v�rdier opdateres.
	\item Systemet g�r i dvale. Bussernes position vil ikke l�ngere holdes opdateret.
	\end{enumerate}
\end{enumerate}
\begin{enumerate}[\text{Undtagelse} C:] 
\item Positions data for busserne kan ikke tilg�s. 
	\begin{enumerate}[1.]
	\item Systemet viser en fejlmeddelelse til brugeren, der beskriver, at det ikke er muligt at indl�se bussernes position.
	\item Kortet vil blive vist, hvor busserne er p� deres senest hentede position.
	\item Brugeren f�r vist en meddelse om at bussernes position ikke l�ngere bliver opdateret.
	\end{enumerate}
\end{enumerate}
\begin{enumerate}[\text{Undtagelse} D:] 
\item Positions data for busserne kan igen tilg�s. 
	\begin{enumerate}[1.]
	\item Systemet forts�tter i normalforl�bet fra punkt 6. Bussernes position holdes igen opdateret.
	\end{enumerate}
\end{enumerate}
