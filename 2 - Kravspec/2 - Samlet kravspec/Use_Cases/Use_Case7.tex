\subsection{Use Case 7: Rediger busruteplan}
\textbf{M�l:}
\\
M�let med denne Use Case er, at kunne rediger en busruteplan.
\\
\\
\textbf{Initiering:}
\\
Administratoren tilkendegiver over for systemet at han �nsker at lave �ndringer i ruteplanen for en bus.
\\
\\
\textbf{Akt�rer og interessenter:}
\\
Prim�re akt�rer:
\begin{itemize}
\item Administrator.
\end{itemize}
\textbf{Antal samtidige forekomster:}
\\
En samtidig forekomst.
\\
\\
\textbf{Ikke funktionelle krav:}
\begin{itemize}
\item TBD.
\end{itemize}
\textbf{Startbetingelser:}
\\
Brugeren skal have administrator rettigheder, for at kunne tilg� denne del af systemet.
\\
\\
\textbf{Slutresultat ved succes:}
\\
�ndringer vil er blevet persisteret.
\\
\\
\textbf{Slutresultat ved undtagelser:}
\\
�ndringer er ikke blevet persisteret.
\\
\\
\textbf{Normalforl�b A: Tilf�jelse af busrute}
	\begin{enumerate}
	\item Administrator tilkendegiver overfor systemet, at der �nskes at tilf�jes en busrute.
	\item Systemet pr�senter administratoren for et v�rkt�j, hvor et tomt kort kan ses. Her her er det muligt at tilf�je en ny rute, med veje og busstoppesteder.
	\item Administratoren tilf�jer en rute.
	\item Administratoren tilkendegiver overfor systemet, at ruten skal gemmes.
	\item Den tilf�jede rute persisteres.
	\end{enumerate}
	\textbf{Normalforl�b B: Fjernelse af busrute}
	\begin{enumerate}
	\item Administrator tilkendegiver overfor systemet, at der �nskes at fjerne en allerede persisteret rute.
	\item Administratoren tilkendegiver overfor systemet, at fjernelsen af ruten skal gemmes.
	\item Systemet fjerner den valgte rute persisteringen.
	\end{enumerate}
	\textbf{Normalforl�b C: �ndring af busrute}
	\begin{enumerate}
	\item Administrator tilkendegiver over for systemet, at der �nskers at �ndre en given busrute.
	\item Et v�rkt�j bliver pr�senteret, med den valgte rute indtegnet. Her kan der tilf�jes eller fjernes veje og busstoppesteder.
	\item Administratoren �ndrer ruten.
	\item Administratoren tilkendegiver over for systemet, at �ndringer skal gemmes.
	\item Systemet persister den �ndrede rute.
	\end{enumerate}
\textbf{Undtagelser}
	\begin{enumerate}[\text{Undtagelse 1}]

	\item Administratoren annullerer �ndringsprocessen, f�r tilf�jelser, fjernelser eller �ndringer er foretaget.
		
		\begin{enumerate}[1.]
		\item Der returneres til administrations-hovedsk�rmen.
		\end{enumerate}
	\end{enumerate}
		\begin{enumerate}[\text{Undtagelse 2}]

	\item Administratoren annullerer �ndringsprocessen, efter tilf�jelser, fjernelser eller �ndringer er foretaget.
		
		\begin{enumerate}[1.]
		\item Administratoren bliver pr�senteret for en meddelelse , hvori der bliver spurgt, om der vil gemmes eller ej.
		\begin{enumerate}[A]
		\item Hvis der ikke �nskes at gemmes, returneres der til administrations-hovedsk�rmen.
		\item Hvis der �nskes at gemmes, bliver �ndringerne persisteret. Herefter returneres der til administrations-hovedsk�rmen.
		\end{enumerate}
		\end{enumerate}
	\end{enumerate}
		\begin{enumerate}[\text{Undtagelse 3}]

	\item Det er ikke muligt at persistere data.
		
		\begin{enumerate}[1.]
		\item Administratoren bliver pr�senteret for en fejlmeddelelse, som informerer om, at det ikke er muligt at persistere �ndret data.
		\item Systemet returner til det sted administratoren arbejdede, hvor de �ndringer han har foretaget, stadig er tilstede.
		\end{enumerate}
	\end{enumerate}
	


